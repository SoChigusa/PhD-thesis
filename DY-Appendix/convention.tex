\documentclass[12pt,twoside,book]{article}

\input{../settings}

\begin{document}

%%%%%%%%%%%%%%%%%%%%%%%%%%%%%%%%%%%%%%%%%%%%%%%%%%%%%%%%%%%%%%%%%%%%%%%%%%%%%%%
\section{Conventions and notations}
\label{sec:convention}
%%%%%%%%%%%%%%%%%%%%%%%%%%%%%%%%%%%%%%%%%%%%%%%%%%%%%%%%%%%%%%%%%%%%%%%%%%%%%%%

\vskip 0.1in

In this appendix, we summarize the conventions and notations used throughout the thesis.

Firstly, we use the natural units with
\begin{align}
  c = \hbar = k_B = 1,
\end{align}
where $c$, $\hbar$, and $k_B$ are the speed of light, the reduced Planck constant, and the Boltzmann constant, respectively.

Our convention of the four-dimensional Lorenzian metric is $g^{\mu \nu} = \mathrm{diag} (1, -1, -1, -1)$.
We sometimes use the Pauli matrices defined as
\begin{align}
  \sigma_1 =
  \begin{pmatrix}
    0 & 1\\
    1 & 0
  \end{pmatrix},
  ~~
  \sigma_2 =
  \begin{pmatrix}
    0 & -i\\
    i & 0
  \end{pmatrix},
  ~~
  \sigma_3 =
  \begin{pmatrix}
    1 & 0\\
    0 & -1
  \end{pmatrix},
\end{align}
with $i$ being the imaginary unit.
The slash on any character denotes the so-called Feynmann slash, defined as $\Slash{p} \equiv p^\mu \gamma_\mu$ with four-by-four gamma matrices given by
\begin{align}
  \gamma^0 =
  \begin{pmatrix}
    \bm{0} & \bm{1}\\
    \bm{1} & \bm{0}
  \end{pmatrix},
  ~~
  \gamma^i =
  \begin{pmatrix}
    \bm{0} & \sigma_i\\
    -\sigma_i & \bm{0}
  \end{pmatrix},
\end{align}
The unique exception of this rule is $\Slash{E}_T$, which is used to denote the missing transverse momentum in hadron collider experiments.

We use the notation $g_1$, $g_2$, and $g_3$ for the gauge coupling constant of the SM $U(1)_Y$, $SU(2)_L$, and $SU(3)_c$ gauge group.
We use the so-called grand unified theory normalization of $g_1$; the corresponding charge assignment is different from the conventional assignment of $U(1)_Y$ (with, \textit{e.g.}, charge $-1/2$ for the left-handed leptons) by a factor of $\sqrt{3/5}$.

\end{document}

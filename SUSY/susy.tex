\documentclass[12pt,twoside,book]{article}

\input{../settings}

\begin{document}

%%%%%%%%%%%%%%%%%%%%%%%%%%%%%%%%%%%%%%%%%%%%%%%%%%%%%%%%%%%%%%%%%%%%%%%%%%%%%%%
\section{Review of supersymmetry}
\label{sec:susy}
%%%%%%%%%%%%%%%%%%%%%%%%%%%%%%%%%%%%%%%%%%%%%%%%%%%%%%%%%%%%%%%%%%%%%%%%%%%%%%%

\vskip 0.1in

\rem{More later}

First example is the MSSM, extension of the SM with the so-called
$\mathcal{N}=1$ supersymmetry (SUSY)~\cite{Wess:320631, Martin:1997ns}
that relates a bosonic particle and a fermionic particle.  The
supersymmetry transformations for a complex scalar $\phi$ and its
``superpartner'' Weyl fermion $\psi$ are defined as
\begin{align}
 \delta \phi = \left( \epsilon \psi \right),
 ~&~
 \delta \phi^{*} = \left( \epsilon^\dagger \psi^\dagger \right),\\
 \delta \psi = -i \left(\sigma^\mu \epsilon^\dagger \right) \partial_\mu \phi,
 ~&~
 \delta \psi^\dagger = i \left(\epsilon \sigma^\mu \right) \partial_\mu \phi^{*},
\end{align}
where $\sigma^\mu \equiv (\bm{1}, \bm{\sigma})$ with $\bm{\sigma}$
being Pauli matrices, while $\epsilon$ is an anti-commuting Weyl
fermionic object that parameterizes the SUSY transformation.  The
summation over the spinor indices is assumed inside each parenthesis.
These transformations, if denoted by operators $\epsilon Q$ and
$\epsilon^\dagger Q^\dagger$, are known to form a closed algebra
\begin{align}
 \left[ Q, Q^\dagger \right] &= 2 i \sigma^\mu \partial_\mu,\\
 \left[ Q, Q \right] &= \left[ Q^\dagger, Q^\dagger \right] = 0,
\end{align}
when fields are on-shell.\footnote{
  In order for the algebra to be closed off-shell, one can introduce a new scalar field $F$ without a kinetic term that is often called as an \textit{auxiliary} field.
  $F$ works as a Lagrange multiplier whose equation of motion \rem{What?}
}

\rem{chiral and vector superfield}

\rem{F-term and D-term potential}


In this appendix, we briefly review the $\mathcal{N}=1$ supersymmetry, which is an essential element of the MSSM explained in Sec.~\ref{sec:mssm}.
Our argument is based on~\cite{Wess:320631, Martin:1997ns}.

First, we show the $\mathcal{N} = 1$ supersymmetry algebra: \cite{Haag:1974qh}
\begin{align}
  \{ Q_\alpha, \bar{Q}_{\dot{\beta}} \} &= 2 \sigma_{\alpha
  \dot{\beta}}^\mu P_\mu,\label{QQbar} \\
  \{ Q_\alpha, Q_\beta \} &= \{ \bar{Q}_{\dot{\alpha}}
  \bar{Q}_{\dot{\beta}} \}  = 0,\label{QQQbarQbar} \\
  [ P_\mu, Q_\alpha ] &= [ P_\mu, \bar{Q}_{\dot{\alpha}} ] =
  0,\label{PQPQbar}\\
  [ P_\mu, P_\nu ] &= 0,\label{PP}
\end{align}
where $\sigma^\mu$ is defined with a unit matrix and Pauli matrices as $\sigma^\mu = (-\textbf{1},\vec{\sigma})$ and $Q_\alpha$, $\bar{Q}_{\dot{\alpha}}$, and $P_\mu$ are generators of two types of supersymmetry and translation, respectively.
Indices $(\alpha,\beta,\dot{\alpha},\dot{\beta})$ run from one to two and denote two-component Weyl spinors.
Indices $(\mu,\nu)$ run from zero to three and denotes the Lorentz four-vector.
Generators in the above algebra generates the maximally possible symmetries of the $S$-matrix including fermionic operators $Q_\alpha$ and $\bar{Q}_{\dot{\alpha}}$ by loosening the assumption on the symmetry in the derivation of the Coleman-Mandula theorem \cite{Coleman:1967ad}.

There are two important features in the representation of the supersymmetry algebra.
(I) All particles in each representation have the same mass.
(II) Bosonic and fermionic degrees of freedom in each representation are the same.
The property (I) is the direct result of the commutation relation Eq.\ (\ref{PQPQbar}).
To prove (II), we define a fermion number operator $N_F$ which has an eigenvalue $0$ for bosonic states and $+1$ for fermionic states.
From this definition, it is a straightforward work to derive the anti-commutation relation $\{(-1)^{N_F},Q\}=0$ and its conjugate.
Then the following calculation for some finite-dimensional representation of generators
\begin{align}
 {\rm Tr} \left[ (-1)^{N_F} \{Q_\alpha, \bar{Q}_{\dot{\beta}}\} \right]
 = {\rm Tr} \left[ \bar{Q}_{\dot{\beta}} \{ (-1)^{N_F}, Q_\alpha \} \right] = 0,
\end{align}
shows, using Eq.\ (\ref{QQbar}), that
\begin{align}
 2{\rm Tr} \left[ (-1)^{N_F} P_{\alpha \dot{\beta}} \right]
 = 2P_{\alpha \dot{\beta}} {\rm Tr} \left[(-1)^{N_F} \right] = 0,
\end{align}
with $P_{\alpha \dot{\beta}} \equiv \sigma^\mu_{\alpha \dot{\beta}} P_\mu$.
The first equality follows from the fact that the four momentum is universal for elements of an irreducible representation.
The last equality is just another expression of the property (II) for some non-zero four-momentum $P_{\alpha \dot{\beta}}$.

To formulate the supersymmetric field theory, it is convenient to consider the superfield, which lives on the extension of the Minkowski space with four fermionic coordinates $\theta^\alpha$ and $\bar{\theta}_{\dot{\alpha}}$, the so-called super-Minkowski space.
In a representation that acts on the super-Minkowski space, a group element corresponding to operators shown above is expressed as
\begin{align}
 G(x,\theta,\bar{\theta}) = \exp \left[ i \left( -x^\mu P_\mu + \theta Q + \bar{\theta} \bar{Q} \right) \right],
\end{align}
where the indices of fermionic objects are contracted.  Then, by
calculating the product of two group elements using Hausdorff's formula,
supersymmetry transformation is found to be a 'translation' in the
super-Minkowski space \cite{Salam:1981xd, Ferrara:1974ac}, denoted as
\begin{align}
 Q_\alpha &= \frac{\partial}{\partial \theta^\alpha} - i\sigma_{\alpha
 \dot{\alpha}}^\mu \bar{\theta}^{\dot{\alpha}} \partial_\mu,\label{Qrep} \\
 \bar{Q}^{\dot{\alpha}} &= \frac{\partial}{\partial
 \bar{\theta}_{\dot{\alpha}}} + i\theta^\alpha
 \sigma_{\alpha\dot{\beta}}^\mu \epsilon^{\dot{\beta} \dot{\alpha}}
 \partial_\mu.
\end{align}
It is a straightforward calculation to check the anti-commutation
relation of these operators, resulting in
\begin{align}
 \{ Q_\alpha, \bar{Q}_{\dot{\beta}} \} &= 2i\sigma_{\alpha
 \dot{\beta}}^\mu \partial_\mu,\label{QQbarrep} \\
 \{ Q_\alpha, Q_\beta \} &= \{ \bar{Q}_{\dot{\alpha}},
 \bar{Q}_{\dot{\beta}} \} = 0,
\end{align}
which agree with Eqs.\ (\ref{QQbar}) and (\ref{QQQbarQbar}) under the
usual definition of $P_\mu \equiv -i\partial_\mu$.\footnote{There is
actually a disagreement of the sign between Eq.\ (\ref{QQbar}) and Eq.\
(\ref{QQbarrep}).  This difference is not essential since it is due to
the fact that a group element $G(X) G(Y)$ ($X,Y \in \RRR^{1,3|4}$)
corresponds to the reversed order translation first induced by $X$ and
second by $Y$.}  In the super-Minkowski space, we can decompose the most
general function called superfield into
\begin{align}
 F(x,\theta,\bar{\theta}) &= \phi(x) + \theta \psi(x) + \bar{\theta}
 \bar{\psi}(x) \notag \\
 &\quad + \theta\theta F(x) + \bar{\theta}\bar{\theta} \bar{F}(x) +
 \theta \sigma^m \bar{\theta} v_m(x) \notag \\
 &\quad + \theta\theta\bar{\theta} \lambda(x) +
 \bar{\theta}\bar{\theta}\theta \bar{\lambda}(x) +
 \theta\theta\bar{\theta}\bar{\theta} D(x),
\end{align}
where all the coefficients are general fields with proper spins under
the Lorentz symmetry that live on the Minkowski space.  Operators
involved in the supersymmetry algebra, $Q,\bar{Q}$ and $P$, naturally
act on the superfield $F(x,\theta,\bar{\theta})$ with above
representations.

Next we impose some constraint on the most general superfield to get
special superfields which possess required properties in order to
consider the supersymmetric extension of the SM.  First, we define
chiral covariant derivatives as
\begin{align}
 D_\alpha &= \frac{\partial}{\partial \theta^\alpha} -
 2i\sigma^\mu_{\alpha \dot{\alpha}} \bar{\theta}^{\dot{\alpha}}
 \frac{\partial}{\partial y^\mu}, \\
 \bar{D}_{\dot{\alpha}} &= - \frac{\partial}{\partial
 \bar{\theta}^{\dot{\alpha}}},
\end{align}
where $y^\mu$ is a redefined bosonic coordinate related to $x^\mu$ as
\begin{align}
 y^\mu \equiv x^\mu + i\bar{\theta} \sigma^\mu \theta.
\end{align}
These derivatives are covariant in the meaning that they satisfy the
relation
\begin{align}
 \{ Q_\alpha, D_\beta \} = \{ \bar{Q}_{\dot{\alpha}}, D_\beta \} = \{
 Q_\alpha, \bar{D}_{\dot{\beta}} \} = \{ \bar{Q}_{\dot{\alpha}},
 \bar{D}_{\dot{\beta}} \} = 0,
\end{align}
and hence following equations are satisfied:
\begin{align}
 (\xi Q + \bar{\xi} \bar{Q}) (D_\beta F(y,\theta,\bar{\theta})) &=
 D_\beta ((\xi Q + \bar{\xi} \bar{Q}) F(y,\theta,\bar{\theta})), \\
 (\xi Q + \bar{\xi} \bar{Q}) (\bar{D}_{\dot{\beta}}
 F(y,\theta,\bar{\theta})) &= \bar{D}_{\dot{\beta}} ((\xi Q +
 \bar{\xi} \bar{Q}) F(y,\theta,\bar{\theta})),
\end{align}
where $\xi$ and $\bar{\xi}$ are fermionic transformation parameters of
supersymmetry.  Using these derivatives, we can define a chiral
superfield $\Phi$ with a constraint
\begin{align}
 \bar{D}_{\dot{\alpha}} \Phi = 0,
\end{align}
or expressed explicitly in terms of component fields,
\begin{align}
 \Phi(y,\theta) &= \phi(y) + \sqrt{2} \theta \psi(y) +
 \theta\theta F(y) \notag \\
 &=  \phi(x) + i\theta \sigma^\mu \bar{\theta} \partial_\mu \phi(x) +
 \frac{1}{4} \theta\theta\bar{\theta}\bar{\theta} \Box \phi(x) \notag \\
 &\quad + \sqrt{2} \theta \psi(x) - \frac{i}{\sqrt{2}} \theta\theta
 \partial_\mu \psi(x) \sigma^\mu \bar{\theta} + \theta\theta F(x),\label{chiralsuperf}
\end{align}
which naturally contains a chiral fermion $\psi$ that is an important
ingredient of the SM.  Since Higgs fields also can be implemented in
this type of multiplet as a lowest component $\phi$, the remaining
piece is the spin one gauge fields $A_\mu$.  They are implemented in
vector superfields defined by a constraint
\begin{align}
 V = \bar{V},
\end{align}
or in terms of component fields,
\begin{align}
 V(x,\theta,\bar{\theta}) &= C(x) + i\theta\chi(x) -
 i\bar{\theta}\bar{\chi}(x) + \frac{i}{2}\theta\theta [M(x) + iN(x)] -
 \frac{i}{2} \bar{\theta}\bar{\theta} [M(x) - iN(x)] \notag \\
 &\quad - \theta \sigma^\mu \bar{\theta} A_\mu(x) +
 i\theta\theta\bar{\theta} \left[ \bar{\lambda}(x) +
 \frac{i}{2}\bar{\sigma}^\mu \partial_\mu \chi(x) \right] -
 i\bar{\theta}\bar{\theta}\theta \left[ \lambda(x) + \frac{i}{2}
 \sigma^\mu \partial_\mu \bar{\chi}(x) \right] \notag \\
 &\quad + \frac{1}{2} \theta\theta\bar{\theta}\bar{\theta}\left[ D(x) +
 \frac{1}{2}\Box C(x)\right],\label{vectorsuperf}
\end{align}
where $\bar{\sigma}^\mu = (-\textbf{1},-\vec{\sigma})$ and component
fields are real scalar fields, Majorana fermion fields, or gauge
fields, depending on the spin under the Lorentz symmetry.  For general
gauge theories with a gauge coupling $g$ and generators $T^{aj}_i$, we
prepare several vector superfields labeled by $a$ and use a
combination $V_i^j = 2gT^{aj}_i V^a$.

Now we have to construct the supersymmetry invariant lagrangian in
terms of chiral and vector superfields.  The required property of the
lagrangian is that its variation under supersymmetry transformations
should be total derivative by the bosonic coordinate.  For example,
from Eqs.\ (\ref{Qrep}) and (\ref{chiralsuperf}), we can calculate for
a chiral superfield $\Phi$ that
\begin{align}
 \left[ (\xi Q + \bar{\xi} \bar{Q}) \Phi \right]_F
 = i\sqrt{2} \bar{\xi} \bar{\sigma}^\mu \partial_\mu \psi,
\end{align}
where $[\tilde{\Phi}]_F$ denotes the $\theta\theta$ component (or
F-term) of a chiral superfield $\tilde{\Phi}$.  Since the above
expression is a total derivative if $\bar{\xi}$ is a global parameter,
we can add a F-term of some chiral superfield to the lagrangian.
Similarly for the vector superfield, we can check from Eq.\
(\ref{vectorsuperf}) that the variation of the
$\theta\theta\bar{\theta}\bar{\theta}$ component (or D-term) is a
total derivative:
\begin{align}
 \left[ (\xi Q + \bar{\xi} \bar{Q}) V \right]_D = \frac{1}{2} \xi
 \sigma^\mu \partial_\mu \left[ \bar{\lambda} + \frac{i}{2}
 \bar{\sigma}^\nu \partial_\nu \chi \right] + \frac{1}{2} \bar{\xi}
 \bar{\sigma}^\mu \partial_\mu \left[ \lambda + \frac{i}{2} \sigma^\nu
 \partial_\nu \bar{\chi}\right].
\end{align}
From this, we conclude that we can also add a D-term of some vector
superfield to the lagrangian.

Using what we have learned above, we can finally construct the
lagrangian for a supersymmetric gauge theory.  The first important
observation is that the D-term of a vector superfield $\bar{\Phi}
\Phi$ contains kinetic terms of the component scalar field $\phi$ and
the chiral fermion field $\psi$.  In fact, it can be easily seen from
Eq.\ (\ref{chiralsuperf}) that
\begin{align}
 \left[ \bar{\Phi} \Phi \right]_D \sim -\partial_\mu \phi^{*}
 \partial^\mu \phi -i \bar{\psi} \bar{\sigma}^\mu \partial_\mu \psi +
 \bar{F}F,
\end{align}
up to surface terms.  The situation is a bit complicated for vector
superfields due to the degree of freedom of the gauge transformation.
As an analogy to the non-supersymmetric gauge theory, we can define a
gauge transformation parameter $\Lambda_i^j \equiv T^{aj}_i \Lambda_a$
using a set of chiral superfields $\Lambda_a$.  Then the
transformation rule for each superfield is written as
\begin{align}
 \Phi' &= e^{-i\Lambda} \Phi,\label{chiralsfgaugetransf} \\
 \bar{\Phi}' &= \bar{\Phi} e^{i\bar{\Lambda}},  \\
 e^{V'} &= e^{-i\bar{\Lambda}} e^V e^{i\Lambda},
 \label{vectorsfgaugetransf}
\end{align}
where we use the matrix form of $V$ and $\Lambda$ defined above.
Thanks to this gauge degree of freedom, we can choose a particular
gauge in which Eq.\ (\ref{vectorsuperf}) is significantly simplified,
\begin{align}
 V_{\rm WZ} (x,\theta,\bar{\theta}) = -\theta \sigma^\mu \bar{\theta}
 A_\mu(x) + i\theta\theta \bar{\theta} \bar{\lambda}(x) -
 i\bar{\theta} \bar{\theta} \theta \lambda(x) + \frac{1}{2}
 \theta\theta \bar{\theta} \bar{\theta} D(x),
\end{align}
where the subscript 'WZ' represents the name of this gauge, the
Wess-Zumino gauge.  Although choosing this gauge breaks supersymmetry,
we can fix one reference frame of $\RRR^{1,3|4}$ at first, and
continue our discussion under the WZ gauge in this frame.  Next we
need an analog of the field strength of the non-supersymmetric gauge
theory which transforms covariantly under the gauge transformation.
The required quantity is a chiral superfield defined as
\begin{align}
 \mathcal{W}_\alpha \equiv -\frac{1}{4} \bar{D} \bar{D} (e^{-V}
 D_\alpha e^V),
\end{align}
where its transformation rule under the gauge symmetry are
\begin{align}
 \mathcal{W}'_\alpha = e^{-i\Lambda} \mathcal{W}_\alpha e^{i\Lambda}.
\end{align}
Then, as is expected, the F-term of the invariant combination ${\rm
Tr}[\mathcal{W}\mathcal{W}]$ contains terms proportional to the
kinetic terms of $A_\mu$ and $\lambda$,
\begin{align}
 \left[ {\rm Tr}[\mathcal{W}\mathcal{W}] \right]_F = 4kg^2 \left[
 -2i\lambda^a \sigma^\mu \nabla_\mu \bar{\lambda}^a - \frac{1}{2}
 F^{a\mu\nu} F^a_{\mu\nu} + D^a D^a + \frac{i}{4} F^a_{\mu\nu}
 F^a_{\rho \sigma} \epsilon^{\mu\nu\rho\sigma} \right],\label{trwawa}
\end{align}
where $k\delta^{ab} \equiv {\rm Tr}[T^a T^b]$ and $\nabla_\mu$ and
$F^a_{\mu\nu}$ are the gauge covariant derivative and the (bosonic)
field strength, respectively.  Note that the standard interaction among
a gauge boson and two fermions is naturally introduced through the
covariant derivative.  For later convenience, we again decompose
$\mathcal{W}_\alpha$ as
\begin{align}
 \mathcal{W}_\alpha = 2g T^a \mathcal{W}^a_\alpha.
\end{align}
Using this decomposition, ${\rm Tr}[\mathcal{W}\mathcal{W}]$ in Eq.\
(\ref{trwawa}) can be deformed as
\begin{align}
 \frac{1}{4kg} {\rm Tr}[\mathcal{W}\mathcal{W}] = \mathcal{W}^a
 \mathcal{W}^a.
\end{align}
Finally we have to comment that the kinetic
term of a chiral superfield should also be gauge invariant.  As is
easily read off from Eqs.\
(\ref{chiralsfgaugetransf})-(\ref{vectorsfgaugetransf}), the
combination $\bar{\Phi} e^V \Phi$ satisfies the invariance.  This
modification naturally introduces gauge interactions of $\phi$ and
$\psi$ as
\begin{align}
 \left[ \bar{\Phi} e^V \Phi \right]_D &\sim -\nabla_\mu \phi^{*}
 \nabla^\mu \phi -i \bar{\psi} \bar{\sigma}^\mu \nabla_\mu \psi +
 F\bar{F} \notag \\
 &\quad -\sqrt{2} g(\phi^{*} T^a \psi) \lambda^a - \sqrt{2} g
 \bar{\lambda}^a (\bar{\psi} T^a \phi) + g(\phi^{*} T^a \phi) D^a,\label{chiralkinetic}
\end{align}
up to surface terms.

\begin{table}[t]
 \begin{center}
  \begin{tabular}{|c|c|c|c||c|c|}
   \hline
   & $\phi$ & $\psi$ & $F$ & bosonic & fermionic \\ \hline
   on-shell & 2 & 2 & 0 & 2 & 2 \\ \hline
   off-shell & 2 & 4 & 2 & 4 & 4 \\ \hline
  \end{tabular} \\ \vspace{5mm}
  \begin{tabular}{|c|c|c|c||c|c|}
   \hline
   & $A_\mu$ & $\lambda$ & $D$ & bosonic & fermionic \\ \hline
   on-shell & 2 & 2 & 0 & 2 & 2 \\ \hline
   off-shell & 3 & 4 & 1 & 4 & 4 \\ \hline
  \end{tabular}
 \end{center}
 \caption{The counting of bosonic and fermionic degrees of freedom in
 the chiral superfield (up) and the vector superfield (down).
 Off-shell, auxiliary fields posses non-zero degree of freedom and
 keep bosonic and fermionic degrees of freedom equal in each
 representation.}
 \label{counting}
\end{table}

In summary, we get the supersymmetric gauge invariant lagrangian of
the form
\begin{align}
 \mathcal{L}_{\rm free} = \frac{1}{4} \left[ \int d^2 \theta\ {\rm Tr}
 [\mathcal{W}^a \mathcal{W}^a] + {\rm c.c.} \right] + \int d^2 \theta
 d^2 \bar{\theta}\ \sum_{i} \bar{\Phi}_i e^V
 \Phi_i,\label{freelagrangian}
\end{align}
where the index $i$ discriminates different chiral superfields and
$\int d^2 \theta$ and $\int d^2 \theta d^2 \bar{\theta}$ are the same
meaning as $[\cdots ]_F$ and $[\cdots ]_D$, respectively, because of
the Grassmann nature of coordinates $\theta$ and $\bar{\theta}$.  In
the lagrangian, component fields $F_i$ and $D^a$ involved in the
chiral and the vector superfield, respectively, are called auxiliary
fields and are needed to make the supersymmetry algebra closed
off-shell (without using equations of motion).  This can be seen from
the counting of bosonic and fermionic degrees of freedom shown in
Tab.\ \ref{counting}.  However, when considering on-shell, we can use equations of
motion for these fields and completely eliminate them.  Then, the
physical degrees of freedom left are chiral fermions $\psi_i$ and
their superpartners $\phi_i$, and gauge bosons $A_\mu^a$ and their
superpartners $\lambda^a$ called gauginos.  Eq.\
(\ref{freelagrangian}) uniquely specifies the form of invariant
kinetic terms and interactions among them.  In addition to this, we
can also add interactions among chiral superfields following
procedures above.  The most general renormalizable interaction is
\begin{align}
 \mathcal{L}_{\rm int} &= \int d^2 \theta\ W[\Phi_i] + {\rm c.c.}, \\
 W[\Phi_i] &= L^i \Phi_i + M^{ij} \Phi_i \Phi_j + y^{ijk} \Phi_i \Phi_j
 \Phi_k,
\end{align}
where $W[\Phi_i]$ is called superpotential.  Of course each term in
the superpotential should be gauge invariant combination of chiral
superfields.  See App.\ \ref{sec:notation} for the superpotential of
the minimal supersymmetric standard model (MSSM) and the notation
used throughout this thesis.

Finally we comment on the breaking of supersymmetry.  Since we have not
found particles with same masses as the SM particles but with spins
different by $1/2$, we have to spontaneously break supersymmetry at some
energy scale.  Although it is not difficult to construct a model in
which global supersymmetry is spontaneously broken by the Higgs
mechanism, there are two problem to be considered.  First problem is
about gaugino masses, which can not appear only with renormalizable
interactions since the most general lagrangian we have constructed above
does not include any interaction among scalar-gaugino-gaugino.  Second
is related to the mass relation
\begin{align}
 {\rm STr}(m^2) \equiv \sum_{j} (-1)^{2j} (2j+1) {\rm Tr} (m_j^2),
\end{align}
where $i=0,1/2,1,\cdots$ labels the spin of particles and ${\rm Tr}
(m_i^2)$ is a sum of mass squared over all particles with spin $i$.
Actually, from the most general renormalizable lagrangian and possibly
non-zero vacuum expectation value (VEV) of scalar particles, we can
calculate the above supertrace in globally supersymmetric models as
\begin{align}
 {\rm STr}(m^2) = -2g{\rm Tr}(T^a) D^a = 0,\label{strmsq}
\end{align}
where the last equality holds because $SU(3)_C$ and $SU(2)_L$ generators
are all traceless, while for $U(1)_Y$ it is equivalent to the
cancellation condition of the mixed $U(1)$-gravitational anomaly.  If we
assume the MSSM sfermions (superpartners of the SM fermions) do not mix
with other scalars and also neglect the mixing between different
flavors, this sum rule should be satisfied for each individual lepton
and quark chiral superfield, resulting for example in
\begin{align}
 m_{\tilde{t}_1}^2 + m_{\tilde{t}_2}^2 = 2 m_t^2,
\end{align}
for top quarks and their superpartners called stops.  This relation
obviously contradicts with the mass lower bounds imposed by stop search
experiments \cite{ATLAS:2016kts, CMS:2016mwj}.  One of the possible
solutions for these problems is to consider a local supersymmetry (or
supergravity).  In this type of model, it is important to consider the
supersymmetry breaking effect expressed by VEVs of auxiliary fields.
Then, gaugino masses are generated by some non-renormalizable
interactions involving a chiral superfield with non-zero F-term VEV at
the tree-level or the loop-level, depending on whether there is any
gauge singlet or not.  Also, in this scenario, the right-handed side of
Eq.\ (\ref{strmsq}) becomes some non-zero constant of $O(m_{3/2}^2)$,
where $m_{3/2}$ denotes the gravitino mass.  Then we do not have to
worry about the second problem related to the mass squared sum rule.

Let us take a closer look at this kind of framework.  The model in which
supersymmetry breaking is mediated by some interaction as weak as the
gravity is often called the Planck scale (gravity) mediation
\cite{Chamseddine:1982jx, Barbieri:1982eh, Ibanez:1982ee, Hall:1983iz,
Ohta:1982wn, Ellis:1982wr, AlvarezGaume:1983gj} and described in the
supergravity (SUGRA) framework.  In this scenario, sfermion and Higgsino
masses are generated from the K${\rm \ddot{a}}$hler potential
\begin{align}
 \mathcal{K} \supset \bar{\Phi} \Phi + c \frac{\bar{X} X}{M_{*}^2}
 \bar{\Phi} \Phi + (c' H_u H_d + {\rm h.c.})\ ,
\end{align}
where $X$ denotes a chiral field with non-zero F-term VEV and $M_{*}$ is
the Planck scale.  The resulting masses are\footnote{Here we use the
vanishing cosmological constant condition.}
\begin{align}
 m_{\rm sfermion}^2 &\sim m_{3/2}^2 + c m_{3/2}^2, \\
 \mu &\sim c' m_{3/2}.
\end{align}
Also, if $X$ is a singlet of the gauge symmetry, gaugino masses are
generated by the superpotential
\begin{align}
 W \supset - c'' \frac{X}{M_{*}} \mathcal{W}^a \mathcal{W}^a,
\end{align}
with
\begin{align}
 m_{\rm gaugino} \sim c'' m_{3/2},
\end{align}
and hence all SUSY breaking masses are of $O(m_{3/2})$.  In particular,
if we further assume common scalar masses, gaugino masses, and scalar
trilinear terms at some high energy scale such as the GUT scale, all
soft SUSY breaking terms are determined by four parameters $(m_0,
m_{1/2}, A_0, b^{1/2})$, which is called the mSUGRA boundary condition
\cite{Nilles:1983ge, Kane:1993td}.  This condition is often used as a
benchmark and we will use it as an example of Quasi-natural or
High-scale SUSY models in Sec.\ \ref{sec:phenomenology}.

On the other hand, if we do not have any gauge singlet in the model, we
acquire gaugino masses and scalar trilinear terms suppressed by
$m_{3/2}/M_{*}$ at the tree-level.  In this case, the leading
contribution to these terms is due to the loop effect called anomaly
mediation \cite{Giudice:1998xp,Randall:1998uk}, about which we will
explain in Sec.\ \ref{sec:AMSB}.  This situation is also called as pure
gravity mediation \cite{Ibe:2006de, Ibe:2011aa, ArkaniHamed:2012gw}.
There is a large hierarchy between sfermion masses and gaugino masses in
this case and the resulting model is what we have denoted as the
Mini-Split model.  Again assuming universal scalar masses at some high
energy scale, soft SUSY breaking terms in this model are determined by
three parameters $(m_0, m_{3/2}, b^{1/2})$.  This is a benchmark of the
Mini-Split model and we will also use it in Sec.\
\ref{sec:phenomenology}.

\bibliographystyle{elsarticle-num}
\bibliography{../phd}

\end{document}

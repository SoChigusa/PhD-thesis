\documentclass[12pt,twoside,book]{article}

\input{../settings}

\begin{document}

%%%%%%%%%%%%%%%%%%%%%%%%%%%%%%%%%%%%%%%%%%%%%%%%%%%%%%%%%%%%%%%%%%%%%%%%%%%%%%%
\section{Review of supersymmetric gauge theory}
\label{sec:susy}
%%%%%%%%%%%%%%%%%%%%%%%%%%%%%%%%%%%%%%%%%%%%%%%%%%%%%%%%%%%%%%%%%%%%%%%%%%%%%%%

\vskip 0.1in

In this appendix, we briefly review the $\mathcal{N}=1$ supersymmetric gauge theory, which is an essential element of the MSSM explained in Sec.~\ref{sec:MSSM}.
Our argument is based on~\cite{Wess:320631, Martin:1997ns}.

First, we show the $\mathcal{N} = 1$ supersymmetry algebra: \cite{Haag:1974qh}
\begin{align}
  \{ Q_\alpha, \bar{Q}_{\dot{\beta}} \} &= 2 \sigma_{\alpha
  \dot{\beta}}^\mu P_\mu,\label{QQbar} \\
  \{ Q_\alpha, Q_\beta \} &= \{ \bar{Q}_{\dot{\alpha}}
  \bar{Q}_{\dot{\beta}} \}  = 0,\label{QQQbarQbar} \\
  [ P_\mu, Q_\alpha ] &= [ P_\mu, \bar{Q}_{\dot{\alpha}} ] =
  0,\label{PQPQbar}\\
  [ P_\mu, P_\nu ] &= 0,\label{PP}
\end{align}
where $\sigma^\mu$ is defined with a unit matrix and Pauli matrices as $\sigma^\mu = (-\textbf{1},\vec{\sigma})$ and $Q_\alpha$, $\bar{Q}_{\dot{\alpha}}$, and $P_\mu$ are generators of two types of supersymmetry and translation, respectively.
Indices $(\alpha,\beta,\dot{\alpha},\dot{\beta})$ run from one to two and denote two-component Weyl spinors.
Indices $(\mu,\nu)$ run from zero to three and denotes the Lorentz four-vector.
Generators in the above algebra generate the maximally possible symmetries of the $S$-matrix including fermionic operators $Q_\alpha$ and $\bar{Q}_{\dot{\alpha}}$ by loosening the assumption on the symmetry in the derivation of the Coleman-Mandula theorem \cite{Coleman:1967ad}.

There are two important features in the representation of the supersymmetry algebra.
(I) All particles in each representation have the same mass.
(II) Bosonic and fermionic degrees of freedom in each representation are the same.
The property (I) is the direct result of the commutation relation Eq.\ (\ref{PQPQbar}).
To prove (II), we define a fermion number operator $N_F$ which has an eigenvalue $0$ for bosonic states and $+1$ for fermionic states.
From this definition, it is a straightforward work to derive the anti-commutation relation $\{(-1)^{N_F},Q\}=0$ and its conjugate.
Then the following calculation for some finite-dimensional representation of generators
\begin{align}
  {\rm Tr} \left[ (-1)^{N_F} \{Q_\alpha, \bar{Q}_{\dot{\beta}}\} \right]
  = {\rm Tr} \left[ \bar{Q}_{\dot{\beta}} \{ (-1)^{N_F}, Q_\alpha \} \right] = 0,
\end{align}
shows, using Eq.\ (\ref{QQbar}), that
\begin{align}
  2{\rm Tr} \left[ (-1)^{N_F} P_{\alpha \dot{\beta}} \right]
  = 2P_{\alpha \dot{\beta}} {\rm Tr} \left[(-1)^{N_F} \right] = 0,
\end{align}
with $P_{\alpha \dot{\beta}} \equiv \sigma^\mu_{\alpha \dot{\beta}} P_\mu$.
The first equality follows from the fact that the four-momentum is universal for elements of an irreducible representation.
The last equality is just another expression of the property (II) for some non-zero four-momentum $P_{\alpha \dot{\beta}}$.

To formulate the supersymmetric field theory, it is convenient to consider the superfield, which lives on the extension of the Minkowski space with four fermionic coordinates $\theta^\alpha$ and $\bar{\theta}_{\dot{\alpha}}$, the so-called super-Minkowski space.
In a representation that acts on the super-Minkowski space, a group element corresponding to operators shown above is expressed as
\begin{align}
 G(x,\theta,\bar{\theta}) = \exp \left[ i \left( -x^\mu P_\mu + \theta Q + \bar{\theta} \bar{Q} \right) \right],
\end{align}
where the indices of fermionic objects are contracted.
Then, by calculating the product of two group elements, supersymmetry transformation is found to be a translation in the super-Minkowski space \cite{Salam:1981xd, Ferrara:1974ac}, expressed as
\begin{align}
  Q_\alpha &= \frac{\partial}{\partial \theta^\alpha} - i\sigma_{\alpha
  \dot{\alpha}}^\mu \bar{\theta}^{\dot{\alpha}} \partial_\mu,\label{Qrep} \\
  \bar{Q}^{\dot{\alpha}} &= \frac{\partial}{\partial \bar{\theta}_{\dot{\alpha}}}
  + i\theta^\alpha \sigma_{\alpha\dot{\beta}}^\mu \epsilon^{\dot{\beta} \dot{\alpha}}
  \partial_\mu.
\end{align}
It is a straightforward task to check these representations satisfy the correct commutation relations with the definition of $P_\mu \equiv -i\partial_\mu$.
In the super-Minkowski space, we can decompose the most general function as
\begin{align}
  F(x,\theta,\bar{\theta}) &= \phi(x) + \theta \psi(x) + \bar{\theta} \bar{\psi}(x)
  \notag \\
  &\quad + \theta\theta F(x) + \bar{\theta}\bar{\theta} \bar{F}(x)
  + \theta \sigma^\mu \bar{\theta} v_\mu (x) \notag \\
  &\quad + \theta\theta\bar{\theta} \lambda(x) + \bar{\theta}\bar{\theta}\theta \bar{\lambda}(x) + \theta\theta\bar{\theta}\bar{\theta} D(x),
\end{align}
where all the coefficients are general fields with proper spins under the Lorentz symmetry.
Operators involved in the supersymmetry algebra, $Q,\bar{Q}$, and $P$, naturally act on the superfield $F(x,\theta,\bar{\theta})$ with the above representations.

Next, we impose some constraint on the above superfield to get special superfields which possess required properties when we consider the supersymmetric extension of the SM.
First, we define chiral covariant derivatives as
\begin{align}
  D_\alpha &= \frac{\partial}{\partial \theta^\alpha} -
  2i\sigma^\mu_{\alpha \dot{\alpha}} \bar{\theta}^{\dot{\alpha}}
  \frac{\partial}{\partial y^\mu}, \\
  \bar{D}_{\dot{\alpha}} &= - \frac{\partial}{\partial \bar{\theta}^{\dot{\alpha}}},
\end{align}
where $y^\mu$ is a redefined bosonic coordinate related to $x^\mu$ as
\begin{align}
  y^\mu \equiv x^\mu + i\bar{\theta} \sigma^\mu \theta.
\end{align}
These derivatives are covariant in the meaning that they satisfy the relations
\begin{align}
  \{ Q_\alpha, D_\beta \} = \{ \bar{Q}_{\dot{\alpha}}, D_\beta \} = \{
  Q_\alpha, \bar{D}_{\dot{\beta}} \} = \{ \bar{Q}_{\dot{\alpha}},
  \bar{D}_{\dot{\beta}} \} = 0,
\end{align}
and also the following equations
\begin{align}
  (\xi Q + \bar{\xi} \bar{Q}) (D_\beta F(y,\theta,\bar{\theta})) &=
  D_\beta ((\xi Q + \bar{\xi} \bar{Q}) F(y,\theta,\bar{\theta})), \\
  (\xi Q + \bar{\xi} \bar{Q}) (\bar{D}_{\dot{\beta}}
  F(y,\theta,\bar{\theta})) &= \bar{D}_{\dot{\beta}} ((\xi Q +
  \bar{\xi} \bar{Q}) F(y,\theta,\bar{\theta})),
\end{align}
where $\xi$ and $\bar{\xi}$ are fermionic transformation parameters of the supersymmetry.
Using these derivatives, we define a chiral superfield $\Phi$ with a constraint
\begin{align}
  \bar{D}_{\dot{\alpha}} \Phi = 0,
\end{align}
or expressed explicitly in terms of component fields,
\begin{align}
  \Phi(x,\theta) &=  \phi(x) + i\theta \sigma^\mu \bar{\theta} \partial_\mu \phi(x) +
  \frac{1}{4} \theta\theta\bar{\theta}\bar{\theta} \Box \phi(x) \notag \\
  &\quad + \sqrt{2} \theta \psi(x) - \frac{i}{\sqrt{2}} \theta\theta
  \partial_\mu \psi(x) \sigma^\mu \bar{\theta} + \theta\theta F(x),\label{chiralsuperf}
\end{align}
which naturally contains a chiral fermion $\psi$ that is an important ingredient of the SM.
Since Higgs fields are also implemented in this type of multiplet as the lowest component $\phi$, the remaining piece is the spin one gauge fields $A_\mu$.
They are implemented in vector superfields defined by a constraint
\begin{align}
  V = \bar{V},
\end{align}
or in terms of component fields,
\begin{align}
  V(x,\theta,\bar{\theta}) &= C(x) + i\theta\chi(x) -
  i\bar{\theta}\bar{\chi}(x) + \frac{i}{2}\theta\theta [M(x) + iN(x)] -
  \frac{i}{2} \bar{\theta}\bar{\theta} [M(x) - iN(x)] \notag \\
  &\quad - \theta \sigma^\mu \bar{\theta} A_\mu(x) +
  i\theta\theta\bar{\theta} \left[ \bar{\lambda}(x) +
  \frac{i}{2}\bar{\sigma}^\mu \partial_\mu \chi(x) \right] -
  i\bar{\theta}\bar{\theta}\theta \left[ \lambda(x) + \frac{i}{2}
  \sigma^\mu \partial_\mu \bar{\chi}(x) \right] \notag \\
  &\quad + \frac{1}{2} \theta\theta\bar{\theta}\bar{\theta}\left[ D(x) +
  \frac{1}{2}\Box C(x)\right], \label{vectorsuperf}
\end{align}
where $\bar{\sigma}^\mu = (-\textbf{1},-\vec{\sigma})$ and component
fields are real scalar fields, Majorana fermion fields, or gauge
fields, depending on the spin under the Lorentz symmetry.  For general
gauge theories with a gauge coupling $g$ and generators $T^a_{i j}$, we
prepare several vector superfields labeled by $a$ and use a
combination $V_{i j} = 2gT^a_{i j} V^a$.

Now we demonstrate the way to construct a Lagrangian invariant under the supersymmetry transformations in terms of chiral and vector superfields.
Firstly, we focus on the $\theta\theta$ component (or the F-term) of a chiral superfield $\tilde{\Phi}$, which will be denoted as $[\Phi]_F$ below, and derive its transformation rule as
\begin{align}
  \left[ (\xi Q + \bar{\xi} \bar{Q}) \Phi \right]_F
  = i\sqrt{2} \bar{\xi} \bar{\sigma}^\mu \partial_\mu \psi.
\end{align}
Since the above expression is a total derivative if $\bar{\xi}$ is a global parameter, we can add the F-term of any chiral superfield to the lagrangian.
Similarly, for the vector superfield $V$, we can check that the transformation of the $\theta\theta\bar{\theta}\bar{\theta}$ component (or the D-term), which will be denoted as $[V]_D$, is a total derivative:
\begin{align}
  \left[ (\xi Q + \bar{\xi} \bar{Q}) V \right]_D = \frac{1}{2} \xi
  \sigma^\mu \partial_\mu \left[ \bar{\lambda} + \frac{i}{2}
  \bar{\sigma}^\nu \partial_\nu \chi \right] + \frac{1}{2} \bar{\xi}
  \bar{\sigma}^\mu \partial_\mu \left[ \lambda + \frac{i}{2} \sigma^\nu
  \partial_\nu \bar{\chi}\right].
\end{align}
Thus, we can also add the D-term of any vector superfield to the lagrangian.

Using what we have learned above, we can finally construct the lagrangian of a supersymmetric gauge theory.
The first important observation is that the D-term of a vector superfield $\bar{\Phi} \Phi$ contains kinetic terms of the component scalar field $\phi$ and the chiral fermion field $\psi$.
We can easily see that
\begin{align}
  \left[ \bar{\Phi} \Phi \right]_D \sim -\partial_\mu \phi^{*}
  \partial^\mu \phi -i \bar{\psi} \bar{\sigma}^\mu \partial_\mu \psi + \bar{F}F,
\end{align}
up to surface terms.
For vector superfields, the degrees of freedom of the gauge transformation require some consideration.
As an analogy to the non-supersymmetric gauge theory, we define gauge transformation parameters $\Lambda_{i j} \equiv T^a_{i j} \Lambda_a$ using a set of chiral superfields $\Lambda_a$.
Then the transformation rule for each superfield is written as
\begin{align}
  \Phi' &= e^{-i\Lambda} \Phi,\label{chiralsfgaugetransf} \\
  \bar{\Phi}' &= \bar{\Phi} e^{i\bar{\Lambda}},  \\
  e^{V'} &= e^{-i\bar{\Lambda}} e^V e^{i\Lambda},
  \label{vectorsfgaugetransf}
\end{align}
where we use the matrix form of $V$ and $\Lambda$ defined above.
Thanks to the gauge degrees of freedom, we can choose a particular gauge in which Eq.\ (\ref{vectorsuperf}) is significantly simplified,
\begin{align}
  V_{\mathrm{WZ}} (x,\theta,\bar{\theta}) = -\theta \sigma^\mu \bar{\theta}
  A_\mu(x) + i\theta\theta \bar{\theta} \bar{\lambda}(x) -
  i\bar{\theta} \bar{\theta} \theta \lambda(x) + \frac{1}{2}
  \theta\theta \bar{\theta} \bar{\theta} D(x),
\end{align}
where the name of the gauge, the Wess-Zumino (WZ) gauge \cite{Wess:1974jb} is represented by the subscript.
Although the gauge fixing breaks supersymmetry, we can fix one reference frame of the super-Minkowski space at first, and continue our discussion under the WZ gauge in this frame.
Next, we need an analog of the field strength of the non-supersymmetric gauge theory that transforms covariantly under the gauge transformation.
The required quantity is a chiral superfield defined as
\begin{align}
  \mathcal{W}_\alpha \equiv -\frac{1}{4} \bar{D} \bar{D} (e^{-V} D_\alpha e^V),
\end{align}
where its transformation rule under the gauge symmetry is
\begin{align}
 \mathcal{W}'_\alpha = e^{-i\Lambda} \mathcal{W}_\alpha e^{i\Lambda}.
\end{align}
The F-term of the invariant combination $\mathrm{Tr}[\mathcal{W}\mathcal{W}]$ contains terms proportional to the kinetic terms of $A_\mu$ and $\lambda$ as
\begin{align}
 \left[ \mathrm{Tr}[\mathcal{W}\mathcal{W}] \right]_F = 4kg^2 \left[
 -2i\lambda^a \sigma^\mu \nabla_\mu \bar{\lambda}^a - \frac{1}{2}
 F^{a\mu\nu} F^a_{\mu\nu} + D^a D^a + \frac{i}{4} F^a_{\mu\nu}
 F^a_{\rho \sigma} \epsilon^{\mu\nu\rho\sigma} \right],\label{trwawa}
\end{align}
where $k\delta^{ab} \equiv {\rm Tr}[T^a T^b]$ and $\nabla_\mu$ and $F^a_{\mu\nu}$ are the gauge covariant derivative and the field strength, respectively.
Note that the standard interaction among a gauge boson and two fermions is naturally introduced through the covariant derivative.
For later convenience, we again decompose $\mathcal{W}_\alpha$ as
\begin{align}
 \mathcal{W}_\alpha = 2g T^a \mathcal{W}^a_\alpha,
\end{align}
with which ${\rm Tr}[\mathcal{W}\mathcal{W}]$ can be deformed as
\begin{align}
  \frac{1}{4kg} {\rm Tr}[\mathcal{W}\mathcal{W}] = \mathcal{W}^a \mathcal{W}^a.
\end{align}
Finally, we have to comment that the kinetic term of a chiral superfield can be deformed to be gauge invariant.
As is easily read off from Eqs.~\eqref{chiralsfgaugetransf}--\eqref{vectorsfgaugetransf}, the combination $\bar{\Phi} e^V \Phi$, instead of $\bar{\Phi} \Phi$, becomes gauge invariant.
This modification naturally introduces gauge interactions of $\phi$ and $\psi$ as
\begin{align}
 \left[ \bar{\Phi} e^V \Phi \right]_D &\sim -\nabla_\mu \phi^{*}
 \nabla^\mu \phi -i \bar{\psi} \bar{\sigma}^\mu \nabla_\mu \psi +
 F\bar{F} \notag \\
 &\quad -\sqrt{2} g(\phi^{*} T^a \psi) \lambda^a - \sqrt{2} g
 \bar{\lambda}^a (\bar{\psi} T^a \phi) + g(\phi^{*} T^a \phi) D^a,\label{chiralkinetic}
\end{align}
up to surface terms.

\begin{table}[t]
  \centering
  \begin{tabular}{c|ccc|cc}
    & $\phi$ & $\psi$ & $F$ & bosonic & fermionic \\ \hline
    on-shell & 2 & 2 & 0 & 2 & 2 \\
    off-shell & 2 & 4 & 2 & 4 & 4 \\
  \end{tabular} \\ \vspace{5mm}
  \begin{tabular}{c|ccc|cc}
    & $A_\mu$ & $\lambda$ & $D$ & bosonic & fermionic \\ \hline
    on-shell & 2 & 2 & 0 & 2 & 2 \\
    off-shell & 3 & 4 & 1 & 4 & 4 \\
  \end{tabular}
  \caption{
    The counting of bosonic and fermionic degrees of freedom in the chiral superfield (up) and the vector superfield (down).
    Off-shell, auxiliary fields possess non-zero degrees of freedom and keep bosonic and fermionic degrees of freedom equal in each representation.
  }
  \label{tab:counting}
\end{table}

In summary, we get the supersymmetric gauge invariant Lagrangian of the form
\begin{align}
 \mathcal{L}_{\rm free} = \frac{1}{4} \left[ \int d^2 \theta\ {\rm Tr}
 [\mathcal{W}^a \mathcal{W}^a] + {\rm c.c.} \right] + \int d^2 \theta
 d^2 \bar{\theta}\ \sum_{i} \bar{\Phi}_i e^V
 \Phi_i,\label{freelagrangian}
\end{align}
where the index $i$ discriminates different chiral superfields and $\int d^2 \theta$ and $\int d^2 \theta d^2 \bar{\theta}$ are the same as $[\cdots ]_F$ and $[\cdots ]_D$, respectively, because of the Grassmann nature of the coordinates $\theta$ and $\bar{\theta}$.
In the lagrangian, component fields $F_i$ and $D^a$ involved in chiral and vector superfields, respectively, are called auxiliary fields and are needed to make the supersymmetry algebra closed off-shell, \textit{i.e.}, without using equations of motion.
This can be seen from the counting of bosonic and fermionic degrees of freedom shown in Table~\ref{tab:counting}.
However, when considering on-shell, we can use equations of motion for these fields and completely eliminate them.
Then, the physical degrees of freedom left are chiral fermions $\psi_i$ and their superpartners $\phi_i$, and gauge bosons $A_\mu^a$ and their superpartners $\lambda^a$ called gauginos.

Eq.~\eqref{freelagrangian} uniquely specifies the form of supersymmetry and gauge invariant kinetic terms and interactions among them.
Besides, we can also add interactions among chiral superfields following the procedures described above.
The most general renormalizable interaction is
\begin{align}
 \mathcal{L}_{\rm int} &= \int d^2 \theta\ W[\Phi_i] + {\rm c.c.}, \label{ref:Fterm} \\
 W[\Phi_i] &= L^i \Phi_i + M^{ij} \Phi_i \Phi_j + y^{ijk} \Phi_i \Phi_j
 \Phi_k,
\end{align}
where $W[\Phi_i]$ is called the superpotential.
Each term in the superpotential should be a gauge invariant combination of chiral superfields.
See, for example, Sec.~\ref{sec:MSSM} for the superpotential of the minimal supersymmetric standard model (MSSM).
Adding Eq.~\eqref{ref:Fterm} to the lagrangian, there are two different origins of the scalar potential in a supersymmetric gauge theory, one from the F-term of the superpotential $W[\Phi_i]$ and the other from the D-term of the gauge invariant combination $\bar{\Phi}_i e^V \Phi$.
They are often called the F-term and D-term potentials, respectively.

% \bibliographystyle{elsarticle-num}
% \bibliography{../phd}

\end{document}

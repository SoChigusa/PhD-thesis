\documentclass[12pt,twoside,book]{article}

\input{../settings}

\begin{document}

%%%%%%%%%%%%%%%%%%%%%%%%%%%%%%%%%%%%%%%%%%%%%%%%%%%%%%%%%%%%%%%%%%%%%%%%%%%%%%%
\section{Review of supersymmetry}
%%%%%%%%%%%%%%%%%%%%%%%%%%%%%%%%%%%%%%%%%%%%%%%%%%%%%%%%%%%%%%%%%%%%%%%%%%%%%%%

In this appendix, we briefly review the $\mathcal{N}=1$ supersymmetry,
which is an essential element of the MSSM reviewed in
Sec.~\ref{sec:mssm}.  Our argument is based on~\cite{Wess:320631,
Martin:1997ns}.

The $\mathcal{N}=1$ supersymmetry is

First example is the MSSM, extension of the SM with the so-called
$\mathcal{N}=1$ supersymmetry (SUSY)~\cite{Wess:320631, Martin:1997ns}
that relates a bosonic particle and a fermionic particle.  The
supersymmetry transformations for a complex scalar $\phi$ and its
``superpartner'' Weyl fermion $\psi$ are defined as
\begin{align}
 \delta \phi = \left( \epsilon \psi \right),
 ~&~
 \delta \phi^{*} = \left( \epsilon^\dagger \psi^\dagger \right),\\
 \delta \psi = -i \left(\sigma^\mu \epsilon^\dagger \right) \partial_\mu \phi,
 ~&~
 \delta \psi^\dagger = i \left(\epsilon \sigma^\mu \right) \partial_\mu \phi^{*},
\end{align}
where $\sigma^\mu \equiv (\bm{1}, \bm{\sigma})$ with $\bm{\sigma}$
being Pauli matrices, while $\epsilon$ is an anti-commuting Weyl
fermionic object that parameterizes the SUSY transformation.  The
summation over the spinor indices is assumed inside each parenthesis.
These transformations, if denoted by operators $\epsilon Q$ and
$\epsilon^\dagger Q^\dagger$, are known to form a closed algebra
\begin{align}
 \left[ Q, Q^\dagger \right] &= 2 i \sigma^\mu \partial_\mu,\\
 \left[ Q, Q \right] &= \left[ Q^\dagger, Q^\dagger \right] = 0,
\end{align}
when fields are on-shell.\footnote{
%%
In order for the algebra to be closed off-shell, one can introduce a
new scalar field $F$ without a kinetic term that is often called as an
\textit{auxiliary} field.  $F$ works as a Lagrange multiplier whose
equation of motion }

\end{document}

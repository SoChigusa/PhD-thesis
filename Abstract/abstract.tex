\documentclass[12pt]{article}

\input{../settings}

\renewcommand{\abstractname}{}

\begin{document}

\begin{titlepage}

\centering

\begin{titlelineskip}

\vskip .5in

{\Large

論文の内容の要旨

}

\vskip .5in

{\Large \bf

  Probing Electroweakly Interacting Massive Particles\\
  with Drell-Yan Process at 100 TeV Colliders\\

  \vskip 5mm

  (100 TeV コライダーに置けるレプトン対生成過程を用いた電弱相互作用を持つ新粒子の間接探索)

}

\vskip 1cm

{\Large

  氏名  千草 颯

}

\end{titlelineskip}

\vspace{0.5in}

\begin{abstract}
  There are many extensions of the standard model that predict the existence of electroweakly interacting massive particles (WIMPs), in particular in the context of the dark matter.
  WIMPs, which may be the dominant component of the dark matter, can be searched for using several different methods, such as the direct and indirect detection of the dark matter and the direct production at collider experiments.
  However, it is known that Higgsino, which is an example of the WIMP contained in the supersymmetric extension of the standard model, is difficult to search for in many cases.
  In this thesis, we provide a way for indirectly studying WIMPs through the precision study of the pair production processes of charged leptons or that of a charged lepton and a neutrino at future $100\,\mathrm{TeV}$ collider experiments.
  It is revealed that this search method is suitable in particular for Higgsino, providing us the $5\sigma$ discovery reach of Higgsino in supersymmetric model with mass up to $850\,\mathrm{GeV}$.
  We also show that this search method provides important and independent information about every kind of WIMP in addition to Higgsino.
  Finally, we also discuss how accurately one can extract the mass, gauge charge, and spin of WIMPs in our method.
\end{abstract}

\end{titlepage}

\thispagestyle{empty}

\end{document}

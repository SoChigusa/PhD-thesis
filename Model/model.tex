\documentclass[12pt,twoside,book]{article}
\usepackage{docmute}

\input{../settings}

\begin{document}

%%%%%%%%%%%%%%%%%%%%%%%%%%%%%%%%%%%%%%%%%%%%%%%%%%%%%%%%%%%%%%%%%%%%%%%%%%%%%%%
\section{Models with WIMPs}
\setcounter{equation}{0}
%%%%%%%%%%%%%%%%%%%%%%%%%%%%%%%%%%%%%%%%%%%%%%%%%%%%%%%%%%%%%%%%%%%%%%%%%%%%%%%

\vskip 0.1in

There are several examples of the models that contain WIMP DM
candidates.  In this section, two of them \rem{Really?} are briefly
reviewed.  \rem{EWIMP and WIMP??}

%%%%%%%%%%%%%%%%%%%%%%%%%%%%%%%%%%%%%%%%%%%%%%%%%%%%%%%%%%%%%%%%%%%%%%%%%%%%%%%
\subsection{Minimally supersymmetric standard model}
%%%%%%%%%%%%%%%%%%%%%%%%%%%%%%%%%%%%%%%%%%%%%%%%%%%%%%%%%%%%%%%%%%%%%%%%%%%%%%%

\begin{figure}[b]
  \centering
  \includegraphics[width=0.6\hsize]{Higgs_mass.pdf}
  \caption{One-loop correction to the Higgs mass from (a) a Weyl fermion $f$ and (b) a complex scalar $S$.}
  \label{fig:Higgs_mass}
\end{figure}

The minimally supersymmetric standard model (MSSM) is the simple extension of the SM with $\mathcal{N} = 1$ supersymmetry (SUSY). \footnote{
  For a brief review of the $\mathcal{N} = 1$ SUSY, see Sec.~\ref{sec:susy}.
}
One of the motivations to introduce SUSY is to solve the so-called hierarchy (or naturalness) problem \cite{Weinberg:1975gm,Gildener:1976ai,Susskind:1978ms} in the SM.
The problem is related to the quantum correction to the SM Higgs boson mass from heavy new physics particles.
For example, we can consider the one-loop correction to the Higgs mass from a Weyl fermion $f$ and a complex scalar $S$ as illustrated in Fig.~\ref{fig:Higgs_mass}.
The corrections to the Higgs mass is given by
\begin{align}
  \Delta m_h^2 &= -\frac{|\lambda_f|^2}{8 \pi^2} \left[
  \Lambda_{\mathrm{UV}}^2 - 2 m_f^2 \ln \left( \frac{\Lambda_{\mathrm{UV}}}{m_f} \right)
  + \cdots \right] & &\mathrm{(fermion)}, \label{eq:delmh_f}\\
  \Delta m_h^2 &= \frac{\lambda_S}{16 \pi^2} \left[
  \Lambda_{\mathrm{UV}}^2 - 2 m_S^2 \ln \left( \frac{\Lambda_{\mathrm{UV}}}{m_S} \right)
  + \cdots \right] & &\mathrm{(scalar)}, \label{eq:delmh_S}
\end{align}
\rem{Check this!}
where $\lambda_f$ and $m_f$ are the Higgs-fermion coupling constant and the fermion mass, respectively, and $\lambda_S$ and $m_S$ are those for the scalar $S$.
We take the cut-off scale of the theory to be $\Lambda_{\mathrm{UV}}$ to regularize the otherwise divergent loop integral and neglect the lower order terms of $\Lambda_{\mathrm{UV}}$.
Eqs.~\eqref{eq:delmh_f} and \eqref{eq:delmh_S} show the quadratic dependence of $\Delta m_H^2$ on $\Lambda_{\mathrm{UV}}$, which means that the Higgs mass is sensitive to the energy scale of the beyond the SM physics.
However, there is at least one extremelly high energy scale physics in the nature, gravity at the Planck scale $M_{\mathrm{pl}} \sim 10^{18 \hyphen 19}\,\mathrm{GeV}$.
By substituting $\Lambda_{\mathrm{UV}} = M_{\mathrm{pl}}$ in Eqs.~\eqref{eq:delmh_f} and \eqref{eq:delmh_S} and assuming $\lambda_f \sim \lambda_S \sim \mathcal{O} (1)$, we notice that orders-of-magnitude fine-tuning is required to obtain the correct Higgs mass $m_h = 125.10\,\mathrm{GeV}$ \cite{Tanabashi:2018oca}, which is unnatural.

SUSY provides a nice solution to this fine-tuning problem.
As is summarized in Appendix~\ref{sec:susy}, \rem{summarize later} each Weyl fermion in a supersymmetric model has two complex scalars with the same mass $m_f = m_S$.
In addition, their coupling constants should have a relationship $|\lambda_f|^2 = \lambda_S$ due to the fact that $\lambda_S$ is a coupling constant in the F-term potential sourced by a superpotential term proportional to $\lambda_f$. \rem{description of F-term and D-term}
By using both equations and summing the corrections \eqref{eq:delmh_f} and \eqref{eq:delmh_S} with factor of two multiplied to the latter, we obtain a result independent of the cut-off scale $\Lambda_{\mathrm{UV}}$ without fine-tuning.
This cancellation is ensured by the so-called non-renormalization theorem. \cite{Salam:1974jj, Grisaru:1979wc}

\begin{table}[t]
  \centering
  \begin{tabular}{c|ccc}
    Notation & $SU(3)_C$ & $SU(2)_L$ & $U(1)_Y$ \\ \hline
    $\hat{Q}_i$ & $\bm{3}$ & $\bm{2}$ & $1/6$ \\
    $\hat{L}_i$ & $\bm{1}$ & $\bm{2}$ & $-1/2$ \\
    $\hat{U}_i$ & $\bar{\bm{3}}$ & $\bm{1}$ & $-2/3$ \\
    $\hat{D}_i$ & $\bar{\bm{3}}$ & $\bm{1}$ & $1/3$ \\
    $\hat{E}_i$ & $\bm{1}$ & $\bm{1}$ & $1$ \\
    $\hat{H}_u$ & $\bm{1}$ & $\bm{2}$ & $1/2$ \\
    $\hat{H}_d$ & $\bm{1}$ & $\bm{2}$ & $-1/2$
  \end{tabular}
  \caption{Notations and quantum numbers of the chiral superfields in the MSSM.}
  \label{tab:mssm_csf}
\end{table}

\begin{table}[t]
  \centering
  \begin{tabular}{c|ccc}
    Notation & $SU(3)_C$ & $SU(2)_L$ & $U(1)_Y$ \\ \hline
    $\hat{g}$ & $\bm{8}$ & $\bm{1}$ & $0$ \\
    $\hat{W}$ & $\bm{1}$ & $\bm{3}$ & $0$ \\
    $\hat{B}$ & $\bm{1}$ & $\bm{1}$ & $0$ \\
  \end{tabular}
  \caption{Notations and quantum numbers of the vector superfields in the MSSM.}
  \label{tab:mssm_vsf}
\end{table}

We now summarize the notations and quantum numbers of the chiral and vector superfields in the MSSM in Table~\ref{tab:mssm_csf} and \ref{tab:mssm_vsf}, respectively.
The supersymmetric part of the MSSM lagrangian is described by the superpotential
\begin{align}
  W = Y_u^{i j} U_i Q_j H_u - Y_d^{i j} D_i Q_j H_d
  - Y_e^{i j} E_i L_j H_d + \mu H_u H_d,
  \label{eq:mssm_sup}
\end{align}
where $i,j=1,2,3$ labels the quark and lepton generation, while $Q, L, U, D, E$ are superfields that contain the left-handed quark, left-handed lepton, right-handed up-type quark, right-handed down-type quark, and right-handed charged lepton, respectively.
In Eq.~\eqref{eq:mssm_sup}, proper contraction of $SU(3)_C$ and $SU(2)_L$ indices is assumed.
Note that two Higgs doublets $H_u$ and $H_d$ with opposite values of $U(1)_Y$ hypercharges are introduced, which is needed to cancel the contributions to the gauge anomaly from Higgs superpartners, Higgsinos.

Since no superpartner of any SM particle is observed yet, SUSY should be broken and superpartners should obtain the SUSY breaking masses.  \rem{ref: boson and fermion obtain equal mass}
The SUSY breaking part of the lagrangian is expressed as
\begin{align}
  \mathcal{L}_{\mathrm{soft}} =&
  -\frac{1}{2} \left( M_3 \g \g + M_2 \W \W + M_1 \B \B + \mathrm{c.c.} \right) \notag \\&
  -\left( A_u^{ij} \U_i \Q_j H_u - A_d^{ij} \D_i \Q_j H_d - A_e^{ij} \E_i \L_j H_d \right) \notag \\&
  -m_Q^{2ij} \Q^\dagger_i \Q_j - m_L^{2ij} \L^\dagger_i \L_j - m_U^{2ij} \U^\dagger_i \U_j
  -m_D^{2ij} \D^\dagger_i \D_j - m_E^{2ij} \E^\dagger_i \E_j \notag \\&
  -m_{H_u}^2 H_u^{*} H_u - m_{H_d}^2 H_d^{*} H_d - \left( b H_u H_d + \mathrm{c.c.} \right),
  \label{eq:mssm_soft}
\end{align}
where the tilde is used to express the superpartner of the SM particle contained in a superfield, while a field without a hat nor tilde denotes the other component.

\rem{Comment on the need of SUSY breaking sector?}

\subsubsection*{Higgs mass in the MSSM}

Under the spontaneously broken SUSY, the cancellation of the quantum correction to the Higgs boson discussed above is not exact.
One obvious consequence of the SUSY breaking in Eqs.~\eqref{eq:delmh_f} and \eqref{eq:delmh_S} is the hierarchy between $m_f$ and $m_S$ that appear in the second term of each contribution.
In the case of the MSSM, the largest contribution comes from the superpartner of the top quark, stop, that have the largest Yukawa coupling with the Higgs boson.

\begin{table}[t]
  \centering
  \begin{tabular}{ccc}
    Value & Description & Reference\\ \hline
    $M_W = 80.384 \pm 0.014\, \mathrm{GeV}$ & Pole mass of the W boson
      & \cite{Group:2012gb,Alcaraz:1016509} \\
    $M_Z = 91.1876 \pm 0.0021\, \mathrm{GeV}$ & Pole mass of the Z boson
      & \cite{Beringer:1900zz} \\
    $M_h = 125.15 \pm 0.24\, \mathrm{GeV}$ & Pole mass of the Higgs
      & \cite{Aad:2013wqa,Chatrchyan:2013mxa} \\
    $M_t = 173.34 \pm 0.82\, \mathrm{GeV}$ & Pole mass of the top quark
      & \cite{ATLAS:2014wva} \\
    $\left( \sqrt{2} G_\mu \right)^{-1/2} = 246.21971 \pm 0.00006\, \mathrm{GeV}$
      & Fermi constant for $\mu$ decay & \cite{Tishchenko:2012ie} \\
    $\alpha_3 (M_Z) = 0.1184 \pm 0.0007$
      & $\overline{\mathrm{MS}}$ $SU(3)_C$ gauge coupling & \cite{Bethke:2012jm}
  \end{tabular}
  \caption{Experimentally measured SM parameters used for the derivation of Eq.~\eqref{eq:lambda_at_top}.}
  \label{tab:SM_param}
\end{table}

When there is a large hierarchy between the SUSY breaking scale $M_S$, which is comparable with stop masses, and the top mass $M_t$, the stop contributions to the Higgs mass contains a large logarithm of the form of $\log \left( M_S^2 / M_t^2 \right)$.
\rem{Consistency with above equations.  Maybe MSbar better}
To resum this large logarithm and obtain a precise result, we have to rely on the renormalization group equation (RGE).
In this framework, the value of the Higgs self coupling $\lambda$ at the electroweak scale is closely related to the Higgs mass.
We assume the SM parameters summarized in Table~\ref{tab:SM_param} and the definition of the SM Higgs potential
\begin{align}
  V(H) = -\frac{m^2}{2} |H|^2 + \lambda |H|^4,
\end{align}
with $H$ being the SM Higgs doublet.
Then, according to \cite{Buttazzo:2013uya}, we obtain the relationship
\footnote{
  Although the values listed in Table~\ref{tab:SM_param} are different from the latest ones given in \cite{Tanabashi:2018oca}, we use older ones because the change in input values may cause the slight change in coefficients of second and third terms of Eq.~\eqref{eq:lambda_at_top}.
  The latest central values of the Higgs and top masses are $M_h = 125.10\,\mathrm{GeV}$ and $M_t = 173.1\,\mathrm{GeV}$, with which we can estimate $\lambda (M_t) = 0.12595$.
}
\begin{align}
  \lambda (M_t) = 0.12604
  + 0.00206 \left( \frac{M_h}{\mathrm{GeV}} - 125.15 \right)
  - 0.00004 \left( \frac{M_t}{\mathrm{GeV}} - 173.34 \right).
  \label{eq:lambda_at_top}
\end{align}

In the MSSM, the value of $\lambda$ at the SUSY breaking scale $M_S$ is given by
\begin{align}
  \lambda (M_S) = \frac{g_1^2 (M_S) + g_2^2 (M_S)}{8} \cos^2 2\beta + \delta \lambda,
  \label{eq:lambda_at_ms}
\end{align}
where $g_1$ and $g_2$ are $U(1)_Y$ and $SU(2)_L$ gauge coupling constants, respectively, while $\beta$ parametrizes the ratio of the vacuum expectation values
\begin{align}
  \frac{\Braket{H_u^0}}{\Braket{H_d^0}} = \tan \beta,
\end{align}
with $H_u^0$ and $H_d^0$ being electromagnetically neutral components of the corresponding Higgs doublets.
In Eq.~\eqref{eq:lambda_at_ms}, the first term shows the tree-level contribution from the D-term potential and $\delta \lambda$ denotes the threshold correction from heavy superpartners.
Once the spectrum of the MSSM particles is fixed, we can evaluate the Higgs self coupling using Eq.~\eqref{eq:lambda_at_ms}, calculate its running according to the RGE, and obtain the prediction for the Higgs mass through Eq.~\eqref{eq:lambda_at_top}.

\rem{Contour plot of $m_h$ in $\tan \beta$ vs. $M_S$ plane here.}
\rem{Simply assume $m_{Q3} = m_{U3}$ and compare minimal and maximal mixings}
\rem{Discussion and confirm that heavy SUSY is motivated}


\subsubsection*{Light Wino in the anomaly mediated SUSY breaking model}

Among many SUSY breaking mechanisms, the anomaly mediated SUSY breaking \cite{Giudice:1998xp, Randall:1998uk} leads to


%%%%%%%%%%%%%%%%%%%%%%%%%%%%%%%%%%%%%%%%%%%%%%%%%%%%%%%%%%%%%%%%%%%%%%%%%%%%%%%
\subsection{Need review}
%%%%%%%%%%%%%%%%%%%%%%%%%%%%%%%%%%%%%%%%%%%%%%%%%%%%%%%%%%%%%%%%%%%%%%%%%%%%%%%

\begin{table}
 \centering
 \begin{tabular}{c|ccc|cc}
  & \multicolumn{3}{c|}{Quntum numbers} & \multicolumn{2}{c}{Masses}\\
  WIMP DM candidate & $SU(2)_L$ & $U(1)_Y$ & Spin & $m_\chi / \mathrm{TeV}$ &
  $\Delta m_\chi / \mathrm{MeV}$ \\
  \hline
  Higgsino & $2$ & $1/2$ & Dirac fermion & 1.1 & 341 \\
  Wino & $3$ & $0$ & Majorana fermion & 2.9 & 166 \\
  5-plet scalar & $5$ & $0$ & real scalar & 9.4 & 166 \\
  5-plet fermion & $5$ & $0$ & Majorana fermion & 10 & 166
 \end{tabular}
 \caption{Table of properties of popular WIMP DM
 candidates~\cite{Farina:2013mla, ArkaniHamed:2006mb, Hisano:2006nn,
 Cirelli:2007xd, Moroi:2013sla, Beneke:2016ync}.  The $SU(2)_L$
 electroweak charge, $U(1)_Y$ hypercharge, spin nature, mass, and mass
 difference compared with a charged component of the multiplet are
 shown.  See Sec.~\ref{???}  \rem{Caution!!} for the details of the last
 column.}  \label{tab_WIMP_property}
\end{table}

\rem{Relationship between $\lambda$ parameter above should be clearer}
WIMPs with mass around or just above the electroweak scale are
theoretically well-motivated in connection with problems of the SM such
as the naturalness problem.  For example, the minimal supersymmetric
extension of the SM (the so-called MSSM) contains several WIMP DM
candidate such as Higgsino and Wino.\footnote{
%%
For a review of the MSSM, see for example~\cite{Martin:1997ns}.
}
%%
Another example is the minimal dark matter (MDM)
model~\cite{Cirelli:2005uq, Cirelli:2007xd, Cirelli:2009uv}, which is a
simple extension of the SM with an $SU(2)_L$ electroweak multiplet such
as a $5$-plet scalar / fermion.  In these models, the stability of the
DM is ensured by the $R$-parity (for the MSSM case) and by high
dimensionality of the operator that describes the decay of the DM (for
the MDM case).  The properties of these WIMP DM candidates are
summarized in Table~\ref{tab_WIMP_property}.  The required masses to
explain the DM relic abundance through the freezeout mechanism are also
shown.  Since the non-relativistic annihilation cross section of
$\mathrm{TeV}$ mass particles is significantly enhanced by the
Sommerfeld enhancement effect~\cite{Hisano:2004ds, Hisano:2006nn}, there
are deviations from the rough estimation formula
Eq.~\eqref{eq_relic_abundance}.  We will return to this point later in
Sec.~\ref{???}.  \rem{Caution!!}  In addition, in the last column there
are mass differences $\Delta m_\chi$ between the DM and its charged
couterpart that will be explained in detail in Sec.~\ref{???}.
\rem{Caution!!}

%%%%%%%%%%%%%%%%%%%%%%%%%%%%%%%%%%%%%%%%%%%%%%%%%%%%%%%%%%%%%%%%%%%%%%%%%%%%%%%
\subsection{Minimal dark matter model}
%%%%%%%%%%%%%%%%%%%%%%%%%%%%%%%%%%%%%%%%%%%%%%%%%%%%%%%%%%%%%%%%%%%%%%%%%%%%%%%



\bibliographystyle{elsarticle-num}
\bibliography{../phd}

\end{document}

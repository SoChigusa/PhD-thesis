\documentclass[12pt,twoside,book]{article}
\usepackage{docmute}

\input{../settings}

\begin{document}

%%%%%%%%%%%%%%%%%%%%%%%%%%%%%%%%%%%%%%%%%%%%%%%%%%%%%%%%%%%%%%%%%%%%%%%%%%%%%%%
\section{Direct collider search of WIMPs}
\setcounter{equation}{0}
%%%%%%%%%%%%%%%%%%%%%%%%%%%%%%%%%%%%%%%%%%%%%%%%%%%%%%%%%%%%%%%%%%%%%%%%%%%%%%%

\vskip 0.1in

In this section, we review the production of $\mathrm{TeV}$-scale WIMPs and search for their signals using the collider experiment.
In particular, we will summarize the current bounds for WIMPs obtained at the large hadron collider (LHC) and future bounds expected at the future planned $100\,\mathrm{TeV}$ colliders such as the hadron option of the future circular collider (FCC-hh) \cite{Benedikt:2651300} and the super proton-proton collider (SPPC) \cite{CEPC-SPPCStudyGroup:2015csa, CEPC-SPPCStudyGroup:2015esa}.
In Sec.~\ref{sec:wimp_production}, we discuss the dominant production processes of WIMPs at a hadron collider.
In Sec.~\ref{sec:disappearing_track} and \rem{???}, we review \rem{two???} different methods for the signal identification, the disappearing track search and mono-jet search \rem{???}, and summarize the current and future bounds.


%%%%%%%%%%%%%%%%%%%%%%%%%%%%%%%%%%%%%%%%%%%%%%%%%%%%%%%%%%%%%%%%%%%%%%%%%%%%%%%
\subsection{WIMP production}
\label{sec:wimp_production}
%%%%%%%%%%%%%%%%%%%%%%%%%%%%%%%%%%%%%%%%%%%%%%%%%%%%%%%%%%%%%%%%%%%%%%%%%%%%%%%

There are two relevant processes both of which significantly contribute to the WIMP production cross section.
The pair production via electroweak interaction is a universal process that can be considered for any WIMP considered in this thesis.
The decay of colored particles may also be efficient particularly for the MSSM.
In this subsection, we will review these two in order.


\subsubsection*{Pair production via electroweak interaction}

\begin{figure}[b]
  \centering
  \includegraphics[width=0.4\hsize]{WIMP_production.pdf}
  \caption{WIMP pair production process at the hadron collider.}
  \label{fig:wimp_production}
\end{figure}

Since all the WIMPs considered here possess non-zero $SU(2)_L$ and $U(1)_Y$ charges, they can be directly produced via electroweak interaction at the hadron collider as shown in Fig.~\ref{fig:wimp_production}.
\footnote{
  All the Feynman diagrams in this thesis are drawn with the public code \texttt{JaxoDraw-2.1} \cite{BINOSI20091709}, which is a graphical user interface that allows users to draw Feynman diagrams intuitively and export them in the \texttt{eps} format with the help of the (modification of) \texttt{axodraw} style file for \LaTeX \cite{VERMASEREN199445}.
  Under the environment of macOS Mojave, it apparently fails to start, but one can still execute it by looking inside the application and start the Java executable file \texttt{jaxodraw-2.1-0.jar} directly.
  We would like to thank the authors for providing the best tools to write the thesis with.
  \rem{Where is the first place of Feynman diagrams?}
}
In the figure, $q^\alpha$ and $q^\beta$ denote the partons (namely, one of quarks or gluon) of the incident protons relevant for the process, while $\chi$ denotes the WIMP and $q$ is the momentum transfer.
Assuming the WIMP to be a $SU(2)_L$ $n$-plet with $U(1)_Y$ charge $Y$ and the mass $m_\chi$, this process is well described by the effective lagrangian
\footnote{
  In this subsection, we neglect the small mass difference among different components in the multiplet $\chi$ described in \ref{sec:disappearing_track}.
  This approximation is valid since the mass difference is by far smaller than $m_\chi$ and has only a tiny effect on the production process.
}
\begin{align}
  \mathcal{L} &= \mathcal{L}_{\mathrm{SM}} + (D^\mu \chi)^\dagger (D_\mu \chi) - m_\chi^2 \chi^\dagger \chi &
  &\text{(complex scalar)}, \label{eq:lag_scalar}\\
  \mathcal{L} &= \mathcal{L}_{\mathrm{SM}} + \bar{\chi} (i \Slash{D} - m_\chi) \chi &
  &\text{(Dirac fermion)}, \label{eq:lag_fermion}
\end{align}
with $\mathcal{L}_{\mathrm{SM}}$ being the SM lagrangian, while the covariant derivative is given by
\begin{align}
  D_\mu \equiv \partial_\mu - i g_2 \Slash{W}^a T_n^a - i g_1 Y \Slash{B},
\end{align}
where $T_n^a$ ($a=1,2,3$) are $n$-dimensional representation matrices of $SU(2)_L$.
Note that when $\chi$ is a real scalar (Majorana fermion) with $Y=0$, the terms with $\chi$ in Eq.~\eqref{eq:lag_scalar} (Eq.~\eqref{eq:lag_fermion}) should be devided by two.

For the calculation, we neglect the effect of the electroweak symmetry breaking, which is valid because we are interested in the high-energy collision with the parton-level center-of-mass (CM) energy $\sqrt{s'} \equiv \sqrt{q^2} \gtrsim \mathrm{TeV}$.
Then, we consider the process in the CM frame and estimate the parton-level differential cross section as
\begin{align}
  \left. \frac{d \sigma_{\alpha \beta}}{d \sqrt{s'} d \Omega} \right|_{\text{CM}}
  &= \frac{C_{\alpha \beta}}{8 s'} \left( 1 - \frac{4 m_\chi^2}{s'} \right)^{3/2} \sin^2 \theta
  & &(\text{complex scalar}) \label{eq:parton_cross_section_scalar} \\
  \left. \frac{d \sigma_{\alpha \beta}}{d \sqrt{s'} d \Omega} \right|_{\text{CM}}
  &= \frac{C_{\alpha \beta}}{4 s'} \sqrt{1 - \frac{4 m_\chi^2}{s'}}
  \left[ 1 + \frac{4 m_\chi^2}{s'} + \left( 1 - \frac{4 m_\chi^2}{s'} \right) \cos^2 \theta \right]
  & &(\text{Dirac fermion}), \label{eq:parton_cross_section_fermion}
\end{align}
where $\theta$ is the angle between the momentum of the initial parton $q_a$ and that of one of the final state WIMPs.
These expressions are valid only when the center of mass energy exceeds the production threshold, $\sqrt{s'} > 2m_\chi$.
Note also that these expressions represent inclusive cross sections, \textit{i.e.}, the total cross section for the production of any component of the WIMP multiplet $\chi$.
The coefficient $C_{\alpha \beta}$ consists of contributions from $U(1)_Y$ and $SU(2)_L$ gauge bosons,
\footnote{
  There is no contribution from the interference term between $U(1)_Y$ and $SU(2)_L$ contributions, since it is proportional to $\mathrm{Tr} (T^a_n) = 0$.
}
\begin{align}
  C_{\alpha \beta} = c_{1 \alpha \beta} Y^2 \alpha_1^2
  + c_{2 \alpha \beta} I(n) \alpha_2^2,
\end{align}
with $I(n)$ being the Dynkin index for the $n$-dimensional representation given by
\begin{align}
  I(n) \equiv \frac{n^3-n}{12}.
  \label{eq:dynkin}
\end{align}
The explicit form of $c_{1 \alpha \beta}$ and $c_{2 \alpha \beta}$, which are sizes of the couplings between partons of our choice and gauge bosons, can be expressed using the $U(1)_Y$ charge for a parton $Y_\alpha$ and the $SU(2)_L$ reducible 13-dimensional representation matrices for partons $T^a_{\alpha \beta}$ as
\begin{align}
  c_{1 \alpha \beta} &= Y_\alpha^2 \delta_{\alpha \beta},\\
  c_{2 \alpha \beta} &= \sum_a \left| T^a_{\alpha \beta} \right|^2.
\end{align}
Recalling that $\alpha_1 < \alpha_2$ and that we often consider the WIMPs with large $n$ and moderate $Y$, the WIMP production cross section grows as $n^3$ for larger multiplets according to Eq.~\eqref{eq:dynkin}.

As is well-known, the initial state of the hadron collider is not the individual partons but two protons.
To obtain the cross section for the two protons initial state, we rely on the parton distribution function (PDF), which expresses the fraction of the partons with some given momentum in each accelerated proton.
Let $f_a (x)$ ($0 < x < 1$) be the PDF for a given parton $a$ inside a proton with momentum $p^\mu$.
$f_a (x)$ can be interpreted as a probability distribution to find the parton $a$ with momentum $x p^\mu$, so we have a relationship
\begin{align}
  \sum_a \int_0^1 dx \, x f_a (x) = 1,
\end{align}
associated with the total momentum conservation, and
\begin{align}
  \int_0^1 dx \, \left[ f_d (x) - f_{\bar{d}} (x) \right] &= 1,\\
  \int_0^1 dx \, \left[ f_u (x) - f_{\bar{u}} (x) \right] &= 2,
\end{align}
from the composition of the proton.
Using the PDF, the cross section for the process of interest at the hadron collider is evaluated as
\begin{align}
  \frac{d \sigma}{d \sqrt{s'} d \Omega} =
  \sum_{a,b} \int_0^1 dx_1 dx_2 \, f_a (x_1) f_b (x_2) \delta \left( s' - s x_1 x_2 \right)
  \left. \frac{d \sigma_{a b}}{d \Omega} \right|_{\text{lab}},
\end{align}
where $\sqrt{s}$ is the CM energy of the proton-proton collision.
Note that the cross section in the integrand is a function of $x_1$ and $x_2$, which is obtained by performing the appropriate Lorentz transformation to $\left. d \sigma_{a b} / d \Omega\, \right|_{\text{CM}}$.
\rem{Comment on factorization scale?}

\begin{figure}[t]
  \centering
  \includegraphics[width=0.8\hsize]{WIMP_production_NLO.pdf}
  \caption{Example of NLO QCD contributions to the WIMP pair production process.}
  \label{fig:WIMP_production_NLO}
\end{figure}

Hadron colliders have several more features related to the strong interaction of quantum chromodynamics (QCD).
Firstly, the next-to-leading order (NLO) QCD contribution to each process is not necessarily negligible.
For the WIMP pair production, the real and virtual emission of a gluon shown in the left and right panels of Fig.~\ref{fig:WIMP_production_NLO}, respectively, give the NLO QCD contributions, which will also be taken into account from now on.
In particular, when the large transverse momentum is important for the phenomenology of our concern, such as the case in Sec.~\rem{???}, the real emission of a gluon with sizable transverse momentum significantly modifies the calculation.
Secondly, all the colored particles in the initial, intermediate, and final states should be accompanied with numbers of soft emissions of gluons, which is the phenomena so-called the parton shower.
In practice, there is a difficulty caused by the partial overlap of the gluon phase space between the one-gluon emission cross section considered as an NLO QCD effect and the same considered as the parton shower.
To avoid this overlap, we often perform the matching procedure, in which we set some merging energy scale by hand and include the contribution to the cross section with gluon energy above (below) the scale only from the NLO QCD (parton shower) calculation.
Finally, the colored particles in the final states should eventually be confined, which is called the hadronization, and observed as some energetic and collimated sprays of hadrons, which as a whole is called jets.

In the following, we perform the numerical calculation, taking account of all the above complexities.
For this purpose, we make use of the Monte Carlo generator \texttt{MadGraph5 aMC@NLO (v2.6.3.2)} \cite{Alwall:2011uj,Alwall:2014hca} with the successive use of \texttt{Pythia8} \cite{Sjostrand:2014zea} for the parton shower, hadronization, and matching and \texttt{Delphes (v3.4.1)} \cite{deFavereau:2013fsa} for the detector simulation, including the definition of jets as observed objects.
We use the so-called MLM-style matching \cite{Mangano:2006rw} with the merging scale of $67.5\,\mathrm{GeV}$ and \texttt{NNPDF2.3QED} with $\alpha_3 (M_Z) = 0.118$ \cite{Ball:2013hta} as a canonical set of PDFs.
% For the renormalization and factorization scales, we adopt the default values of MadGraph5 aMC@NLO, \textit{i.e.}, the central $m^2_T$ scale after $k_T$-clustering of the event.

\begin{table}[t]
  \centering
  \begin{tabular}{c|cccc}
    WIMP name & Higgsino & Wino & $5$-plet Majorana fermion & $5$-plet real scalar \\ \hline
    $\sigma_{\mathrm{LO}}$ $[\mathrm{fb}]$ & 15 & 52 & \rem{???} & \rem{???} \\
    $\sigma_{\mathrm{NLO}}$ $[\mathrm{fb}]$ & 17 & 60 & \rem{???} & \rem{???} \\ \hline
    $K$-factor & 1.15 & 1.15 & &
  \end{tabular}
  \caption{
    Table of pair production cross sections of several types of WIMPs.
    The CM energy $\sqrt{s} = 100\,\mathrm{TeV}$ is assumed and WIMP masses are set to be $1\,\mathrm{TeV}$.
  }
  \label{tab:cross_section_WIMPs}
\end{table}

In Table~\ref{tab:cross_section_WIMPs}, we list the production cross sections of various WIMPs via a weak gauge boson exchange at a $\sqrt{s} = 100\,\mathrm{TeV}$ hadron collider.
As for the WIMP mass, we use the common value $m = 1\,\mathrm{TeV}$ to compare the cross sections among different choice of quantum numbers.
$\sigma_{\mathrm{LO}}$ and $\sigma_{\mathrm{NLO}}$ denote the production cross sections without and with the NLO QCD correction, respectively, while the last line is the so-called $K$-factor defined as $K = \sigma_{\mathrm{NLO}} / \sigma_{\mathrm{LO}}$.
From the table, by paying attention to the factor two difference in degrees of freedom between the Dirac (Higgsino) and Majorana (Wino and $5$-plet) fermions, we can roughly see the dependence of the cross section on the $SU(2)_L$ charge $\sigma \propto n^3$.
\rem{Cross section to neutral Higgsino seems missing}

\begin{table}[t]
  \centering
  \begin{tabular}{c|cccc}
    Wino mass $\mathrm{[TeV]}$ & 1.0 & 1.5 & 2.0 & 2.9 \\ \hline
    $\sigma_{\mathrm{LO}}$ $[\mathrm{fb}]$ & 52 & 12 & 4.0 & 0.86\\
    $\sigma_{\mathrm{NLO}}$ $[\mathrm{fb}]$ & 60 & 15 & 4.7 & 1.0 \\ \hline
    $K$-factor & 1.15 & 1.20 & 1.19 & 1.21
  \end{tabular}
  \caption{
    Table of pair production cross sections of Wino with several choice of masses.
    The CM energy $\sqrt{s} = 100\,\mathrm{TeV}$ is assumed.
  }
  \label{tab:cross_section_Wino_mass}
\end{table}

In Table \ref{tab:cross_section_Wino_mass}, we also show the mass dependence of the Wino pair production cross section.
For heavier mass, wider range of $\sqrt{s'}$ is below the production threshold $2 m_{\chi}$ or accompanied with a small suppression factor $(1-4 m_\chi^2 / s')^{1/2}$ as shown in Eq.~\eqref{eq:parton_cross_section_fermion}, and the cross section becomes significantly smaller.
However, values in the tables still denote that plenty of well-motivated WIMP DM candidates, such as $1\,\mathrm{TeV}$ Higgsino and $3\,\mathrm{TeV}$ Wino, are produced at, for example, the $3\,\mathrm{ab}^{-1}$ option of the FCC-hh.

\begin{figure}[t]
  \centering
  \includegraphics[width=0.5\hsize]{invariant_mass.pdf}
  \caption{
    Histogram of the $\sqrt{s'}$ distribution, taking $1\,\mathrm{TeV}$ Higgsino as an example.
    $\sqrt{s} = 100\,\mathrm{TeV}$ and $\mathcal{L} = 3\,\mathrm{ab}^{-1}$ are assumed.}
  \label{fig:invariant_mass}
\end{figure}

In Fig.~\ref{fig:invariant_mass}, we show the $\sqrt{s'}$ distribution for the pair production process of the $m_\chi = 1\,\mathrm{TeV}$ Higgsino.
We assume the setup $\sqrt{s} = 100\,\mathrm{TeV}$ and the integrated luminosity $\mathcal{L} = 3\,\mathrm{ab}^{-1}$.
At around $\sqrt{s'} \sim 2 m_\chi$, we clearly see the production threshold and the suppression effect $\sigma \propto (1-4 m_\chi^2 / s')^{1/2}$.
On the other hand, when $\sqrt{s'}$ becomes much larger than $2m_\chi$, we can see the correct behavior of the cross section, which decreses as $\sigma \propto (\sqrt{s'})^{-3}$ as Eq.~\eqref{eq:parton_cross_section_fermion} indicates.
Note that these properties are universal among several processes, including the dominant contribution \rem{Correct?} to the gluino pair production through the $s$-channel gluon exchange disscused in the next subsection, and the lepton pair production through via an electroweak gauge boson that is the main topics in Sec.~\rem{???}.

\rem{Histogram of angular dependence}

\rem{Is angular dependence affected by the Lorentz boost?}


\subsubsection*{Decay of colored particles}

In hadron colliders, particles with color charges have far more chance to be produced than non-colored particles.
When we consider the split SUSY or the anomaly mediation model reviewed in Sec.~\ref{sec:MSSM}, gluino tends to be relatively light, whose decay produces WIMPs.
Without fine-tuning of Higgsino and gaugino masses, gluino lifetime is sufficiently short and only its decay products are observed by the detectors.
Since all the SUSY particles finally decay into the LSP as described in Sec.~\ref{sec:MSSM}, the gluino production cross section can effectively be counted as the production cross section of WIMPs in these models.

\begin{table}[t]
  \centering
  \begin{tabular}{c|ccc}
    gluino mass $\mathrm{[TeV]}$ & 6.0 & 7.0 & 8.0 \\ \hline
    $\sigma(p p \to \tilde{g} \tilde{g})\, \mathrm{[fb]}$ & 7.9 & 2.7 & 1.0
  \end{tabular}
  \caption{Gluino pair production cross section at $\sqrt{s} = 100\,\mathrm{TeV}$.}
  \label{tab:gluino_pair}
\end{table}

Keeping the R-parity conservation in our mind, the dominant process accompanied with gluinos in these models is the gluino pair production.
In Table \ref{tab:gluino_pair}, we summarize the gluino pair production cross section for various gluino masses at $\sqrt{s} = 100\,\mathrm{TeV}$, taken from \cite{Asai:2019wst}.
The calculation is again performed using \texttt{MadGraph5 aMC@NLO} and only the LO QCD processes are considered.
The values in the table show that the gluino pair production process, dependeing on its mass, may give much larger cross section for the WIMP production than the purely electroweak processes described above.

\rem{Comment on AMSB $m_{3/2}$ and $L$ for the table?}


%%%%%%%%%%%%%%%%%%%%%%%%%%%%%%%%%%%%%%%%%%%%%%%%%%%%%%%%%%%%%%%%%%%%%%%%%%%%%%%
\subsection{Disappearing track search}
\label{sec:disappearing_track}
%%%%%%%%%%%%%%%%%%%%%%%%%%%%%%%%%%%%%%%%%%%%%%%%%%%%%%%%%%%%%%%%%%%%%%%%%%%%%%%

In the last section, we have checked the possibility that a large number of WIMPs are produced at hadron colliders.
On the other hand, the detection of produced WIMPs is not a straight-forward task, because there are huge background events with many charged and/or colored particles.
To reduce the background events and obtain the best possible reach for WIMPs, we consider several methods using typical properties for the WIMP signals, one of which is the disappering track signal described here.

As also mentioned in Sec.~\rem{DM??}, the spontaneous breaking of the electroweak symmetry leads to the mass splitting among an $SU(2)_L$ multiplet, leaving the charge neutral component as the lightest one.
As a result, the charged components of a multiplet, if produced, are unstable and eventually decay into the neutral component.
However, since the mass splitting is so small in many cases that the typical flight length of the charged components is comparable to the detector size.
Such long-lived charged particles, which travel for a few $\mathrm{cm}$ and then decay into an invisible counterpart, can be detected as charged tracks disappearing at the middle.
They are very characteristic signals and can be used as the most efficient discriminator between the SM background and the WIMP signals.
In this section, we will describe what we have summarized above in more detail.

\subsubsection*{Mass splitting among an $SU(2)_L$ multiplet}

First, we consider the mass splitting caused by the spontaneous breakdown of the electroweak symmetry.
In particular, we start with the tree-level propagation of heavy particles, such as the SUSY particles other than the LSP, or another unknown particles.
After integrating out all the heavy particles other than the SM particles and the light WIMP, we may obtain operators of the form of $\mathcal{O} = M_{i j} \chi_i \chi_j$, where $\chi$ denotes the WIMP and $i$ is the $SU(2)_L$ index.
This operator causes the mass splitting only when $M_{i j}$ transforms non-trivially under the $SU(2)_L$ symmetry.
Then, we can explicitlly construct the lowest dimensional operator among those relevant for the mass splitting.
For Higgsino,
\begin{align}
  \mathcal{O} = \frac{1}{\Lambda} (\bar{\chi} H^{*}) (H \chi),
  \label{eq:Higgsino_mass_splitting}
\end{align}
where $\chi = (\tilde{H}_u, -i \sigma_2 \tilde{H}_d^{*})^t$, $\Phi$ is the SM Higgs doublet with $Y = 1/2$, $\Lambda$ is the cut-off scale of the effective theory, \textit{i.e.}, the typical mass scale of the relevant heavy particles, and the parenthesis denotes the $SU(2)_L$ invariant product of fundamental representations.
Similarly, for Wino, \cite{Gherghetta:1999sw}
\begin{align}
  \mathcal{O} = \frac{1}{\Lambda^3} (H^\dagger \sigma^a H) (H^\dagger \sigma^b H) \tilde{W}^a \tilde{W}^b,
  \label{eq:Wino_mass_splitting}
\end{align}
A simple implication of this observation is that, for multiplets with large $n$, there are suppression factors that keep the tree-level mass splitting small.
For Wino, the suppression is of $\mathcal{O} (M_W^4 / \Lambda^3)$, which yields a splitting smaller than $10\,\mathrm{MeV}$ for heavy paritlces with a few $\mathrm{TeV}$ masses.
For fermionic MDMs with $n \gtrsim 5$, a similarly small mass splitting at the tree-level is expected.
\footnote{
  For scalar MDMs, there is another renormalizable operator that generates a mass splitting
  \begin{align*}
    \mathcal{O} = - \lambda_H \left( \chi^{*} \sigma^a \chi \right) \left( H^\dagger \sigma^a H \right).
  \end{align*}
  Unless $\lambda_H$ is sufficiently small, the disappearing track search cannot be applied because of the too large mass splitting.
}
This is the main reason why the loop correction plays more important role for the mass splitting of Wino and MDMs.

The situation is different for Higgsino because of the much less drastic suppression factor of $\mathcal{O} (M_W^2 / \Lambda)$, which generates $\mathcal{O} (100)\,\mathrm{MeV}$ mass splitting for $\Lambda \lesssim \mathrm{10}\,\mathrm{TeV}$.
\footnote{
  For the order estimation of the mass splitting, we have taken account of the size of the coupling constants omitted in Eq.~\eqref{eq:Higgsino_mass_splitting}, using the rough estimation $g_1^2 \sim g_2^2 \sim 1/10$.
}
In fact, in models like the split SUSY, the mixing between Higgsino and heavier gauginos can generate the large mass splitting among the Higgsino components.
As a result, neutral components that originally forms a Dirac fermion splits into two Majorana fermions with mass difference $\Delta m_0$, and the charged components also become heavier than the lighter neutral component by $\Delta m_{+}^{\mathrm{(tree)}}$.
According to \cite{Fukuda:2017jmk}, their approximate expressions are given by
\begin{align}
  \Delta m_0 &\simeq \frac{M_W^2}{g_2^2} \left( \frac{g_1^2}{M_1} + \frac{g_2^2}{M_2} \right),\\
  \Delta m_{+}^{\mathrm{(tree)}} &\simeq \frac{M_W^2}{2 g_2^2} \left[
  \left( \frac{g_1^2}{M_1} + \frac{g_2^2}{M_2} \right)
  + \mathrm{sgn} (\mu) \sin 2\beta \left( \frac{g_1^2}{M_1} - \frac{g_2^2}{M_2} \right) \right],
\end{align}
assuming the CP invariance for simplicity.
Note that the results agrees with the previous order estimation with $\Lambda \sim M_1$, $M_2$.

Next we consider the loop correction to the WIMP masses.
When the loop is composed of heavy particles, the effective operator that causes the mass splitting again becomes the same as above, which is now associated with a small loop factor.
Thus, the largest contribution comes from the gauge boson - WIMP loop.
For the charged componets of Higgsino, the one-loop result is known: \cite{Fukuda:2017jmk}
\begin{align}
  \Delta m_{+}^{\mathrm{(rad)}} \simeq \frac{1}{2} \alpha_2 M_Z \sin^2 \theta_W
  \left( 1 - \frac{3 M_Z}{2\pi m_\chi} \right)
  \sim 355\,\mathrm{MeV} \left( 1 - \frac{3 M_Z}{2\pi m_\chi} \right),
\end{align}
with $\theta_W$ being the Weinberg angle, which gives $\Delta m_{+}^{\mathrm{(rad)}} \simeq 341\,\mathrm{MeV}$ for $m_\chi = 1.1\,\mathrm{TeV}$ and may be comparable to $\Delta m_{+}^{\mathrm{(tree)}}$.
On the other hand, for Wino, we have the two-loop result \cite{Ibe:2012sx}
\newcommand{\logmchi}{\left( \log \frac{m_\chi}{\mathrm{GeV}} \right)}
\begin{align}
  \frac{\Delta m}{\mathrm{MeV}} =
  &-413.315 + 305.383 \logmchi - 60.8831 \logmchi^2\\
  &+ 5.41948 \logmchi^3 - 0.181509 \logmchi^4,
\end{align}
which exhibits $\Delta m \simeq 165\,\mathrm{MeV}$ for $m_\chi = 2.9\,\mathrm{TeV}$.
For the MDM, there are neutral, singly charged, doubly charged, and so on, components.
Among them, the neutral and singly charged components have the smallest mass difference, which is the most important for the collider search, of $\Delta m \simeq 166\, \mathrm{MeV}$ \cite{Cirelli:2005uq}.


\subsubsection*{Lifetime of charged components}



\rem{Histogram of surviving probability}

\rem{Histogram of beta distribution before / after 10cm cut}


%%%%%%%%%%%%%%%%%%%%%%%%%%%%%%%%%%%%%%%%%%%%%%%%%%%%%%%%%%%%%%%%%%%%%%%%%%%%%%%
\subsection{Soft lepton search}
\label{sec:disappearing_track}
%%%%%%%%%%%%%%%%%%%%%%%%%%%%%%%%%%%%%%%%%%%%%%%%%%%%%%%%%%%%%%%%%%%%%%%%%%%%%%%

\rem{If possible}


%%%%%%%%%%%%%%%%%%%%%%%%%%%%%%%%%%%%%%%%%%%%%%%%%%%%%%%%%%%%%%%%%%%%%%%%%%%%%%%
\subsection{Mono-jet search}
\label{sec:disappearing_track}
%%%%%%%%%%%%%%%%%%%%%%%%%%%%%%%%%%%%%%%%%%%%%%%%%%%%%%%%%%%%%%%%%%%%%%%%%%%%%%%

\rem{For Higgsino search, cite} \cite{Baer:2014cua}.


\bibliographystyle{elsarticle-num}
\bibliography{../phd}

\end{document}
